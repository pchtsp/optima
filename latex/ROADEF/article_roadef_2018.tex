\documentclass{roadef}
\usepackage{amsmath}

% \usepackage{fontspec}
% This command is to use simple quotes inside math expressions:
\newcommand{\mq}[1] {`#1\textrm'}

\begin{document}


% Le titre du papier
\title{Maintenance planning on French military aircraft operations}

% Le titre court
\def\shorttitle{Titre court}

\author{Franco Peschiera\inst{1}, Alain Ha�t\inst{1}, Olga Batta�a\inst{1}, Nicolas Dupin\inst{1}}


% Les affiliations (par ordre croissant des num?os d'affiliation) s?ar?s par \and
\institute{
Institut Sup�rieur de l'A�ronautique et de l'Espace \\
\email{\{franco.peschiera,alain.hait,olga.battaia,nicolas.dupin\}@isae-supaero.fr}
}


\maketitle
\thispagestyle{empty}

\keywords{optimization, planning, military, maintenance}


\section{Introduction}

    The Military Flight and Maintenance Planning is a problem consisting on the assignment of missions and maintenances tasks to military aircraft. It is a variant of the Flight and Maintenance Planning problem where flights are not modeled geographically but they are assumed to include a round-trip to the base. It also includes different objectives and constraints.

    Although the former has been studied in literature in \cite{Cho2011}, \cite{Kozanidis2008}, \cite{Verhoeff2015} and \cite{Chastellux2017}, it has not received as much attention as the latter.

    % First of all, the objective function's motivation is on reducing the load on the maintenance facilities and guaranteeing the availability of the fleet instead of reducing the cost. Also, the maintenance operations take considerably longer and the rules that govern the maintenances are different.

    % The temporal scope of the problem is also bigger, making the normal planning horizon bigger and each period bigger too: months instead of days. On the other hand, the physical scope, while present in the commercial planning, is not present in the military one: aircraft are assumed to return to the origin every time.

    % Finally, there are some particular constraints that are specific for this specific problem that have not been found in other works. For example: the possibility to temporally stock the aircraft, the treatment of missions as a complex tasks with requirements and not as an aggregate periodic demand among others.


\section{The problem}

    The problem consists in the planning of the maintenance operations of military aircraft while complying with scheduled missions. In order to generalize the formulation, from now on aircraft will be named "resources" and missions will be named "tasks".

    % \subsection{Tasks}
    % \label{def:task}

    Each $v \in \mathcal{V}$ task needs to be done in a planning horizon divided into $t \in \mathcal{T}$ periods of the same length. Each task has a given starting an ending time that cannot be changed. They also require a certain quantity of assigned resources $a \in \mathcal{A}$ each period $t$, defined by $rr_{v}$ and, if assigned, it uses them for a quantity of time in each period, defined by $rh_{v}$.

    There is a set $cd_{v} \subset \mathcal{A}$ that lists the resources that can be assigned to each task.

    % The assignment of a resource to a task is not decided for the whole duration of the task. After a minimum amount of time ($minTIME_v$), a resource can be freed and exchanged for another one, even if the task it is assigned to has not finished. The total number of resources being used at any given time in a specific task can never be less than the required $REQ_{vt}$.

    % \subsection{Resources}
    % \label{def:resource}

    Resources suffer from wear and tear and so require a recurrent preventive maintenance in order to be in a correct state. These maintenances need to take place before the first of two things occurs: a given amount of elapsed time $me$ has passed since the previous maintenance or a given amount of usage time $mu$ has passed since the previous maintenance. 

    % We will call "remaining elapsed time" (or $ret$) to the amount of time that needs to pass in order for the resource to need a maintenance at any given period.

    % We will call "remaining usage time" (or $rut$) to the amount of time that the resource needs to be used in order for it to need a maintenance at any given period.

    % Additionally, after an absolute amount of time and/or usage ($AET$ or $AUT$), the resource becomes obsolete. There is no way to reverse this process.

    % Each resources starts the planning period with a specific status given by:

    % \begin{itemize}
    %     \item remaining usage time.
    %     \item remaining elapsed time.
    %     % \item remaining absolute elapsed time.
    %     % \item remaining absolute usage time.
    %     % \item LastMaintenanceType (see \ref{def:maint}).
    % \end{itemize}

    % Finally, resources are organized into families or groups. Each resource inside in a family or group shares the same types of maintenances and, usually, the same kind of tasks, among other information.

    % \paragraph{Resource's states}
    % \label{def:res-state}

    The following are the possible $s \in \mathcal{S}$ states for a resource to be in any given time: Assigned to a task ('V'), Receiving maintenance ('M'), Available ('A'), Needing maintenance ('N').

    % \begin{itemize}
    %     \item Assigned to a task (see \ref{def:task}).
    %     \item Under maintenance (see \ref{def:maint}).
    %     % \item Under storage (see \ref{def:sto}).
    %     \item Waiting maintenances.
    %     % \item Obsolete.
    %     \item Available (none of the above).
    % \end{itemize}

    % \subsection{Maintenances}
    % \label{def:maint}

    % Maintenances are the process by which resources that have reached a limit in their usage (but not their absolute limit) can return to a state where they can continue to be used in tasks.

    % Each maintenance belongs to a specific type of maintenance $m \in \mathcal{M}$. These types of maintenance differentiate between each other by having different characteristics.

    These maintenances take an amount of time equal to $md$ periods and cannot be interrupted. The number of resources under maintenance in any given period is limited to a maximum capacity which depends on each period and is represented by $c_t$.
    % The state of the resource after exiting the maintenance period is of "as good as new" in terms of the two indicators that decide the maintenance.
    
    % In other words, after receiving a maintenance, a resource restores its remaining elapsed time and remaining usage time back to their max values $me$ and $mu$ respectively.

    % Not all maintenances are the same for any given resource. Each resources follows a sequence of maintenances. In other words, the second maintenance of a given resource is different in nature from the first maintenance for that same resource. Examples of differences on consecutive maintenances for a given resource are the duration ($mTIME_m$) it takes or whether it restores storage capacity (see \ref{def:sto}).

    % \subsection{Storage}
    % \label{def:sto}

    % We have already explained that even if a resource is not being used, it still needs to have a maintenance after a given amount of time (govern by the elapsed time $elapsedTIME$). In order to avoid this problem, the resource can be put into a "storage" state.

    % A resource in this states has to be kept for a minimum time of $sMin$ periods. While in this state it cannot receive maintenance or be tasked any task.

    % Every resource has the capacity to be stored, although it is limited. This capacity (measured in a number of periods $sCap$) is expended every time the resource is stored. In order for a resource to reset its remaining storage capacity, it needs to received a specific maintenance (see \ref{def:maint}).

    % \subsection{Objectives}
    
    % The are multiple objectives that need to be taken into account. 

    % Given that the creation of new tasks and the duration of maintenance times are considered stochastic, the basic goal is to maximize the robustness of the planning. 

    % Given the limited amount of maintenance capacity and its cost, another goal is to smooth as much as possible the maintenance tasks over the planning horizon.

    % \begin{enumerate}
    %     \item Maximize the minimum number of available resources (see \ref{def:res-state}) in any given period.
    %     \item Minimize the maximum number of resources under maintenance in any given period.
    % \end{enumerate}

\section{The model}

    % \subsection{Decisions}

    % \subsection{Parameters}

    % \begin{tabular}{p{10mm}lp{110mm}}
    % $c_t$            & : & number of resources that can be in maintenance in period $t$. \\
    % $cd_v$           & : & resources that satisfy task's $v$ needs. \\
    % $rh_{vt}$        & : & time consumed of any resource when assigned to task $v$ in period $t$. \\
    % $rr_{vt}$        & : & number or resources to assign to task $v$ in period $t$. \\            
    % $mu$             & : & maximum used time for each resource between maintenances. \\    
    % $me$             & : & maximum elapsed time for each resource between maintenances. \\    
    % $md$             & : & maintenance duration, in periods. \\        
    % $rut\_i_a$       & : & remaining used time for resource $a$ at the beginning of horizon. \\        
    % $ret\_i_a$       & : & remaining elapsed time for resource $a$ at the beginning of horizon. \\
    % $state\_i_{at}$  & : & some $s \in \mathcal{S}$ in case the resource $a$ has a fixed state. \\
    % \end{tabular}
    % \bigskip

    \subsection{Variables}

    The decisions involve, in essence, the management of the previously defined resources. We can summarize the main decisions the following way: if resource $a$ is in state $s$ in each period $t$ and if resource $a$ is assigned to task $v$ in period $t$.

    \begin{tabular}{p{15mm}lp{105mm}}
        $task_{avt}$        & : & 1 if resource $a \in cd_v$ is assignigned task $v$ in period $t$, 0 otherwise. \\  
        $state_{ast}$       & : & 1 if resource $a$ is assigned state $s$ in period $t$, 0 otherwise. \\  
        $start_{at}$        & : & 1 if resource $a$ starts maintenance in period $t$, 0 otherwise. \\
        $ret_{at}$          & : & remaining elapsed time (integer) for resource $a$ after period $t$. \\  
        $rut_{at}$          & : & remaining usage time (integer) for remainingsource $a$ after period $t$. \\  
        $used_{at}$         & : & time (integer) used by resource $a$ in period $t$. \\
        $min\_avail$        & : & min number (integer) of available resources in any period. \\
        $max\_maint$        & : & max number (integer) of resources in maintenance in any period. \\
    \end{tabular}
    % \bigskip

    \subsection{Constraints}

    The objective is to both minimize the maximum number of maintenances and maximize the minimal number of available aircraft:

    \begin{align}
        & \text{Min}\, (max\_maint) \\
        & \text{Max}\, (min\_avail)
    \end{align}
        
    \begin{align}
        %%%%%%%%%%%%%%%%%%%
        % MAIN VARS
        %%%%%%%%%%%%%%%%%%%
        % maximum capacity:
        & \sum_{a \in \mathcal{A}} state_{ast} \leq c_{t}
                & t \in \mathcal{T}, s = \mq{M} \label{eq:capacity}\\
        % minimum resources
        & \sum_{a \in \mathcal{A}} task_{avt} \geq rr_{t}
                & t \in \mathcal{T}, v \in \mathcal{V},  \label{eq:taskres}\\
        % max one task per period
        & \sum_{v \in \mathcal{V}} task_{avt} \leq 1
                & t \in \mathcal{T}, a \in \mathcal{A} \label{eq:maxtasks}\\
        % if assigned to task: in working state
        & state_{ast} \geq task_{avt} 
                & \hspace{10mm} t \in \mathcal{T}, a \in \mathcal{A}, v \in \mathcal{V}, s = \mq{V} \label{eq:state1}\\
        % if no working state, no assigned to any task
        & state_{ast} + task_{avt} \leq 1
                & \hspace{10mm} t \in \mathcal{T}, a \in \mathcal{A}, v \in \mathcal{V}, s \neq \mq{V} \label{eq:state3}\\
        % only one state. If we take out the 'do nothing' state, this could be \leq
        & \sum_{s \in \mathcal{S}} state_{ast} = 1 
                & \hspace{10mm} t \in \mathcal{T}, a \in \mathcal{A} \label{eq:state2}
    \end{align}

    % These constraints are the the main constraints of the model.

    The limit in the number of resources assigned to task and the number of resources under maintenance that can be done in any given month are modeled via \ref{eq:taskres} and \ref{eq:capacity} respectively.
    Constraint \ref{eq:maxtasks} limits the number of assigned tasks for a resource to one.

    Constraint \ref{eq:state1} and \ref{eq:state3} tie states with the assignment to a task. Resources have only one state at each period and this is guaranteed by \ref{eq:state2}.
        
    \begin{align}
        %%%%%%%%%%%%%%%%%%%
        % BALANCES
        %%%%%%%%%%%%%%%%%%%
        % definition of consumed resource hours:
        & used_{at} \geq \sum_{v \in \mathcal{V}} task_{avt} \times rh_{v}
                & a \in \mathcal{A}, t \in \mathcal{T} \label{eq:usage}\\
        % remaining used time
        & rut_{at} \leq (rut_{at-1} - used_{at}) + (mu \times start_{at})
                & t \in \mathcal{T}, a \in \mathcal{A} \label{eq:balance1}\\
        % remaining elapsed time. TODO: improve this constraint like the maintenance state duration
        & ret_{at} \leq (ret_{at-1} - 1) + (me \times start_{at})
                & t \in \mathcal{T}, a \in \mathcal{A} \label{eq:balance2}
    \end{align}
    
    % These constraints calculate the balances of hours for each resource.

    The definition of the consumption of hours for each resource at each period is forced by \ref{eq:usage}. This is then used in \ref{eq:balance1} to get the remaining usage time at each period for each resource. Similarly, the \ref{eq:balance2} calculates the remaining elapsed time at each period and resource.        

    \begin{align}
        %%%%%%%%%%%%%%%%%%%
        % STATES
        %%%%%%%%%%%%%%%%%%%
        % if start maintenance, at least X periods
        & state_{ast_2} \geq start_{at_1} 
                & \hspace{10mm} (t_1, t_2) \in \mathcal{T}, t_1 \leq t_2 \leq t_1 + md - 1, a \in \mathcal{A}, s = \mq{M} \label{eq:start1}\\
        % if maintenance doesn't start, we cannot have maintenance:
        & \sum_{t_1 \in \mathcal{T}} start_{at_1} \geq state_{ast_2} 
                & \hspace{10mm} t_2 \in \mathcal{T}, t_1 \leq t_2 \leq t_1 + md - 1, a \in \mathcal{A}, s = \mq{M} \label{eq:start2}
    \end{align}

    % These constraints involve the states assignments to resources.

    The start of each maintenance with the maintenance state is done by \ref{eq:start1} and \ref{eq:start2}. This way resources need to have started a maintenance to receive it and have to have maintenance for a minimum period after having starting it.

    \begin{align}
        %%%%%%%%%%%%%%%%%%%
        % OBJECTIVE
        %%%%%%%%%%%%%%%%%%%
        % min available
        & min\_avail \leq \sum_{a} state_{ast}
                & \hspace{10mm} t \in \mathcal{T}, a \in \mathcal{A}, s = \mq{A} \label{eq:available}\\
        % max maintenance
        & max\_maint \geq \sum_{a} state_{ast}
                & \hspace{10mm} t \in \mathcal{T}, a \in \mathcal{A}, s = \mq{M} \label{eq:maintenance}
    \end{align}

    % These constraints define the possible objectives.

    The minimum number of available resources in any period is controlled by \ref{eq:available}.

    The definition of the maximum number of resources under maintenance in any period is done by \ref{eq:maintenance}.

    % \begin{align}
    %     %%%%%%%%%%%%%%%%%%%
    %     % HORIZON
    %     %%%%%%%%%%%%%%%%%%%
    %     % TODO: initialize maintenance states too!
    %     % rut initialize
    %     & rut_{at} = rut\_ini_a
    %             & \hspace{10mm} t = 0, a \in \mathcal{A} \label{eq:remaining1}\\
    %     % ret initialize
    %     & ret_{at} = ret\_ini_a
    %             & \hspace{10mm} t = 0, a \in \mathcal{A} \label{eq:remaining2}\\
    %     % fixed states during planning period
    %     & state_{ast} = 1
    %             & \hspace{10mm} t \in \mathcal{T}, a \in \mathcal{A}; s \in \mathcal{S}\, | if\, state\_ini_{at} = s \label{eq:fixedstates}
    % \end{align}

    % % These constraints feed the relevant historic data for resources.

    % The last period's remaining used and elapsed time are filled by \ref{eq:remaining1} and \ref{eq:remaining2} respectively.

    % In order to fix the states to their planned state we need \ref{eq:fixedstates}. This is specially useful for the beginning of the planning period, where we could have maintenance states already decided for some resources.


\section{Conclusions et perspectives}
    
    The present modelization is a new way to formulate the Military Flight Planning Model. Unlike previous work, it assigns aircraft to missions, instead of just using an aggregate demand. This is useful for later improving the model by including minimum durations for mission assignment and other more complex constraints.


% La bibliographie

\bibliographystyle{plain}
\selectlanguage{french}
\bibliography{MFMP}
% \begin{thebibliography}{2}
% \bibitem{toth02}
% Paolo Toth and Daniele Vigo.
% \newblock \emph{The Vehicle Routing Problem}.
% \newblock Monographs on Discrete Mathematics and Applications. Society for Industrial and Applied Mathematics, 2002.
% \bibitem{kirkpatrick83}
% Scott Kirkpatrick, C~Daniel Gelatt, and Mario~P Vecchi.
% \newblock Optimization by simmulated annealing.
% \newblock \emph{science}, 220\penalty0 (4598):\penalty0 671--680, 1983.


% \end{thebibliography}


\end{document}
