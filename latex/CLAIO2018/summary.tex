A specific Flight and Maintenance Planning problem is addressed. In this problem, preventive maintenance operations are scheduled for military aircraft along with the assignment of regular, pre-scheduled, missions.

More generally, a series of $j \in \mathcal{J}$ tasks are planned along a horizon divided into $t \in \mathcal{T}$ periods. Each period usually represents a month. Since all tasks are already scheduled, each task has periods $T_j \subset \mathcal{T}$ in which they will be realized. Each task requires a certain number $r_{j}$ of resources $i \in \mathcal{I}$ which it employs for a time duration defined by $h_j$ hours at each period. Set $I_{j} \subset \mathcal{I}$ lists the resources that can be assigned to each task.

Resources require recurrent preventive maintenance operations since the realization of tasks diminish their remaining usage time. A maintenance operation takes exactly $m$ periods and cannot be interrupted. It assigns to the resource exactly $H$ units of remaining usage time. The remaining usage time not used before the maintenance operation is lost.

Work has already been done in military aircraft planning by \cite{Kozanidis2008, Cho2011}, 

There are several objectives to consider. Reducing the maximum amount of resources under maintenance; maximizing the availability of resources; reducing the total number of maintenance operations.