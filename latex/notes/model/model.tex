\documentclass[a4paper,onecolumn,fleqn]{article}
\textwidth 17cm \textheight 247mm
\topmargin -4mm
\hoffset -9mm \voffset -14mm
%\setlength{\topmargin}{-0.8truecm}
\setlength{\oddsidemargin}{0.3cm}
\setlength{\evensidemargin}{0.3cm}
\setlength{\columnsep}{8mm}
\setlength{\parindent}{0mm}
\setlength{\parskip}{-2.0ex}
\setlength{\mathindent}{0mm}
\flushbottom


\usepackage{epsfig}
%\usepackage{timesnew}
% \usepackage{harvard}
% \usepackage[]{natbib}
\usepackage{amsmath}
\usepackage{booktabs}

\usepackage[latin1]{inputenc}
% \usepackage{aeguill}
\usepackage[english]{babel}
\usepackage{caption}
\usepackage{pgfgantt}


\setlength{\parskip}{2ex} \pagestyle{myheadings}

\renewcommand{\thepage}{}
\renewcommand{\refname}{REFERENCES}



\makeatletter
\renewcommand\section{\@startsection{section}{1}{\z@}%
                       {-6\p@ \@plus -0\p@ \@minus -0\p@}%
                       {2\p@ \@plus 0\p@ \@minus 0\p@}%
                       {\normalsize\textbf}}

\renewcommand\subsection{\@startsection{subsection}{2}{\z@}%
                       {-6\p@ \@plus -0\p@ \@minus -0\p@}%
                       {2\p@ \@plus 0\p@ \@minus 0\p@}%
                       {\normalsize\textbf}}

\renewcommand\subsubsection{\@startsection{subsubsection}{3}{\z@}%
                       {-6\p@ \@plus -0\p@ \@minus -0\p@}%
                       {1\p@ \@plus 0\p@ \@minus 0\p@}%
                       {\normalsize\itshape\bfseries}}
\makeatother


\begin{document}

\section{Mathematical formulation}
  \label{sec:model}

The following model provides a tight MIP formulation that solves the Military Flight and Maintenance Problem. Maintenance operations will be referred as "checks". Missions are assumed to require a constant amount of flight hours per period and a constant amount of aircraft per period, when active. Maintenance capacity is constant overall the planning horizon.

\subsection{Input data}
    \subsubsection{Basic sets}

        \begin{tabular}{p{15mm}p{140mm}}
            $i \in \mathcal{I}$     &  aircraft. Between 15 and 60. \\
            $t \in \mathcal{T}$     &  time periods included in the planning horizon. Between 90 and 120 months. \\
            $j \in \mathcal{J}$     &  missions. There are between 1 and 4 active missions at any given time. \\
        \end{tabular}

    \subsubsection{Auxiliary sets}

        \begin{tabular}{p{15mm}p{140mm}}
            $k \in \mathcal{K}$     &  cluster of aircraft that share the same functionality. \\
            $n \in \mathcal{N}$     &  maintenance cycles for aircraft. Between 2 and 3. \\
            $s \in \mathcal{S}$     &  slots for elastic variables. \\
        \end{tabular}

    \subsubsection{Mission parameters}

        \begin{tabular}{p{15mm}p{125mm}p{15mm}}
            $H_j$             & amount of flight hours required per period by mission $j$. & hours. \\
            $R_j$             & number of aircraft required per period by mission $j$. & aircraft. \\
            $MT^{min}_j$      & minimum number of consecutive periods an aircraft can be assigned to mission $j$. & periods. \\
            $MT^{max}_j$      & maximum number of consecutive periods an aircraft can be assigned to mission $j$. & periods. \\
            $U^{min}$         & default aircraft flight hours if it is not assigned to any mission nor in maintenance.& hours \\
            $M$               & check duration in number of periods. & periods. \\
        \end{tabular}

    \subsubsection{Maintenance parameters}

        \begin{tabular}{p{15mm}p{125mm}p{15mm}}
            $C^{max}$         & maximum number of simultaneous checks. & aircraft. \\
            $E^{min}$         & minimum number of periods between two consecutive checks. & periods. \\
            $E^{max}$         & maximum number of periods between two consecutive checks. & periods. \\
            $H^{max}$         & remaining flight hours for any aircraft after a check. & hours. \\
        \end{tabular}

    \subsubsection{Fleet parameters}

        \begin{tabular}{p{15mm}p{125mm}p{15mm}}
            $N_t$               & number of aircraft known to be in maintenance in period $t$. & aircraft. \\
            $N^{Clust}_{kt}$    & number of aircraft in cluster $k$ known to be in maintenance in period $t$. & aircraft. \\
            $A^{Clust}_{kt}$    & maximum number of aircraft in cluster $k$ that can be simultaneously in maintenance in period $t$. & aircraft. \\
            $H^{Clust}_{kt}$    & maximum number of mean accumulated flight hours for cluster $k$ at maintenance cycle $n$. & hours. \\
            $Rft^{Init}_i$      & remaining flight hours for aircraft $i$ at the start of the planning horizon.  & hours.  \\
            $Rct^{Init}_i$      & remaining calendar time until reaching $E^{max}$ for aircraft $i$ at the start of the planning horizon.  & periods. \\
        \end{tabular}

    \subsubsection{Objective function weights}

        \begin{tabular}{p{15mm}p{125mm}p{15mm}}
            $PA_s$   & unit penalty for violating serviceability constraint in slot $s$. & \\
            $PH_s$   & unit penalty for violating sustainability constraint in slot $s$. & \\
            $PC_s$           & unit penalty for violating capacity constraint in slot $s$. & \\
        \end{tabular}

    \subsubsection{Parametric sets}

        \begin{tabular}{p{15mm}p{140mm}}
            $t \in \mathcal{TJ}_j$     &  time periods $t \in \mathcal{T}$ in which mission $j$ is active. \\
            $j \in \mathcal{JT}_t$    &  missions $j \in \mathcal{J}$ to be realized in period $t$. \\
            $i \in \mathcal{IJ}_j$     &  aircraft $i \in \mathcal{I}$ that can be assigned to mission $j$. \\
            $i \in \mathcal{IK}_k$     &  aircraft $i \in \mathcal{I}$ that are included in cluster $k$. One aircraft can belong to more than one cluster. \\
            $j \in \mathcal{JI}_i$     &  missions $j \in \mathcal{J}$ for which aircraft $i$ can be used. \\
            $i \in \mathcal{A}^{Init}_j$  & aircraft $i \in \mathcal{I}$ that have mission $j$ pre-assigned in the previous period to the start of the planning horizon. \\
        \end{tabular}

    \subsubsection{Time-related parametric sets}

        Several intermediate sets have been defined based on the input data in order to simplify constraint formulation.

        \begin{tabular}{p{30mm}p{125mm}}
            $t' \in \mathcal{T}^s_t$ & time periods $t' \in \mathcal{T}$ during which a check that ends in $t'$ is active, i.e. $t' \in \{ \max{\{1, t - M+1\}},  ..., {t}\}$ (figure \ref{fig:gantt_windows}a). \\
            % $\mathcal{T}^S_t$ &  time periods $t' \in \mathcal{T}$ such that $t' \in \{ {t},  ..., \min{\{|\mathcal{T}|, t + M-1\}}\}$ (figure \ref{fig:gantt_windows}a). \\
            % $t' \in \mathcal{T}^m_t$ &  time periods $t' \in \mathcal{T}$ such that $t' \in \{ {t}, ..., \min{\{|\mathcal{T}|, t + M + E^{min}-1\}}\}$ (figure \ref{fig:gantt_windows}a). \\
            $t' \in \mathcal{T}^M_t$ &  time periods $t' \leq \mathcal{T}$ during which a second check can start, given that the first check started in $t$, i.e. $t' \in \{ t + M + E^{min}-1 , ...,  t + M + E^{max}-1 \}$ (figure \ref{fig:gantt_windows}a). \\
            $(t_1, t_2) \in \mathcal{T}\mathcal{T}\mathcal{T}_{t}$ & pairs of time periods $t_1 \in \mathcal{T}, t_2 \in \mathcal{T}^M_{t_1}$ when a check can start such that it is active in $t$, i.e. $t_1 \in \mathcal{T}^s_{t} \lor t_2 \in \mathcal{T}^s_{t}$  (figure \ref{fig:gantt_windows}b). \\
            $(j, t, t') \in \mathcal{J}\mathcal{T}\mathcal{T}_{it_1t_2}$ & all possible assignments for mission $j$ in between two checks starting at $t_1$ and $t_2$. Formally: triplet composed of mission $j \in \mathcal{JI}_i$ and $t, t' \in \mathcal{TJ}_j$ such that $t' \ge t + MT^{min}_j-1$ and $t \ge t_1 + M$ and $t' < t_2$  (figure \ref{fig:gantt_windows}b).\\
            % $t \in \mathcal{T}^{m_{Init}}_i$ &  time periods $t \in \mathcal{T}$ such that $t \in \{ 1, ..., \max{\{0, Rct^{Init}_i - E^{max} + E^{min} \}}\}$ (figure \ref{fig:gantt_windows}c). \\
            $t \in \mathcal{T}^{M_{Init}}_i$ & possible time periods $t \in \mathcal{T}$ during which aircraft $i$ can have its first check, i.e. $t \in \{ \max{\{0, Rct^{Init}_i - E^{max} + E^{min} \}} , ...,  Rct^{Init}_i \}$ (figure \ref{fig:gantt_windows}c). \\
            % $t' \in \mathcal{T}^{MT}_{jt}$ & time periods $t' \in \mathcal{T}$ such that $t' \in \{ \max{\{1, t - MT^{min}_j\}},  ..., {t}\}$ (figure \ref{fig:gantt_windows}d). \\
            $(t_1, t_2) \in \mathcal{T}\mathcal{T}\mathcal{J}_{jt}$ & pairs of time periods $t_1, t_2 \in \mathcal{TJ}_j$ when a mission $j$ active during $t$ can be assigned, i.e. $t_2 \ge t_1 + MT^{min}_j-1$ and $t \in \{t_1, ..., t_2\}$ (figure \ref{fig:gantt_windows}d). \\
        \end{tabular}

    \begin{figure}
        \centering
        % \includegraphics[width=\linewidth]{img/gantt_timewindows_all.png}
        \begin{tikzpicture}
            % Same month implies one month
% so i need to take one month after summing the amount of months of the duration

\definecolor{color1}{HTML}{FFFFCC}
\definecolor{color2}{HTML}{FFEDA0}
\definecolor{color3}{HTML}{ff0000}
\definecolor{color4}{HTML}{00ff00}
\newcommand\Dganttbar[5]{
    \ganttbar[#5]{#1}{#3}{#4}\ganttbar[inline,bar label font=\tiny\bfseries, #5]{#2}{#3}{#4}
}
\begin{ganttchart}[
    expand chart=\textwidth,
    hgrid,
    vgrid,
    time slot format=isodate,
    time slot unit=month
]{2017-08-01}{2024-06-01}
    \gantttitlecalendar{year} \\
    \Dganttbar{(a)}{M}{2018-03-01}{2018-08-01}{bar/.append style={fill=color1}}
    \Dganttbar{(a)}{M}{2023-07-01}{2023-12-01}{bar/.append style={fill=color1}}
    \ganttvrule[]{$t$}{2018-02-01}
    \ganttvrule[]{$t'$}{2018-08-01}
    \ganttvrule[]{\hspace{25pt} $\mathcal{T}_{t'}^s$}{2018-02-01}
    \ganttvrule[
    vrule label node/.append style={anchor=north west}
    ]{\hspace{50pt} $\mathcal{T}_t^m$}{2018-08-01}
    \ganttvrule[
    vrule label node/.append style={anchor=north west}
    ]{\hspace{50pt} $\mathcal{T}_t^M$}{2021-02-01}
    \ganttvrule[]{}{2023-08-01}
\end{ganttchart}

        \end{tikzpicture}
        \begin{tikzpicture}
            % Same month implies one month
% so i need to take one month after summing the amount of months of the duration

\definecolor{color1}{HTML}{FFFFCC}
\definecolor{color2}{HTML}{FFEDA0}
\definecolor{color3}{HTML}{ff0000}
\definecolor{color4}{HTML}{00ff00}
\newcommand\Dganttbar[5]{
    \ganttbar[#5]{#1}{#3}{#4}\ganttbar[inline,bar label font=\tiny\bfseries, #5]{#2}{#3}{#4}
}
\begin{ganttchart}[
    expand chart=\textwidth,
    hgrid,
    vgrid,
    time slot format=isodate,
    time slot unit=month
]{2017-08-01}{2024-06-01}
    % \gantttitlecalendar{year} \\
    \Dganttbar{(b)}{M}{2018-03-01}{2018-08-01}{bar/.append style={fill=color1}}
    \Dganttbar{(b)}{M}{2023-07-01}{2023-12-01}{bar/.append style={fill=color1}}
    \ganttvrule[]{$t_1$}{2018-02-01}
    \ganttvrule[]{$t_2$}{2023-06-01}
    \ganttvrule[]{\hspace{-30pt} $\mathcal{T}_A$}{2018-08-01}
    \ganttvrule[]{\hspace{-30pt}$\mathcal{T}_B$}{2023-12-01}
    % \ganttvrule[
    % vrule label node/.append style={anchor=north west}
    % ]{\hspace{50pt} $\mathcal{T}_t^M$}{2021-02-01}
    % \ganttvrule[]{}{2023-08-01}
\end{ganttchart}

        \end{tikzpicture}
        \begin{tikzpicture}
            \definecolor{color1}{HTML}{FFFFCC}
\definecolor{color2}{HTML}{FFEDA0}
\definecolor{color3}{HTML}{ff0000}
\definecolor{color4}{HTML}{00ff00}
\newcommand\Dganttbar[5]{
    \ganttbar[#5]{#1}{#3}{#4}\ganttbar[inline,bar label font=\tiny\bfseries, #5]{#2}{#3}{#4}
}
\begin{ganttchart}[
    expand chart=\textwidth,
    hgrid,
    vgrid,
    time slot format=isodate,
    time slot unit=month
]{2017-08-01}{2024-06-01}
    % \gantttitlecalendar{year} \\
    \Dganttbar{(c)}{M}{2020-10-01}{2021-03-01}{bar/.append style={fill=color1}}
    % \Dganttbar{(b)}{M}{2020-10-01}{2021-03-01}{bar/.append style={fill=color1}}
    \ganttvrule[]{}{2019-08-01}
    \ganttvrule[]{\hspace{-120pt}$\mathcal{T}^{M_{Init}}_i$}{2021-04-01}
    \ganttvrule[
        vrule/.style={dotted, line width=0.7pt}
    ]{$t_1$}{2020-09-01}
    \ganttvrule[
        vrule/.style={dotted, line width=0.7pt}
    ]{$t_2$}{2024-06-01}
    % {draw=black, fill=green!50}
    % dashed, line width=1pt, very thick, blue
    % \ganttvrule[
    % vrule label node/.append style={anchor=north west}
    % ]{}{2020-02-01}
\end{ganttchart}

        \end{tikzpicture}
        % \begin{tikzpicture}
        %     \definecolor{color1}{HTML}{FFFFCC}
\definecolor{color2}{HTML}{FFEDA0}
\definecolor{color3}{HTML}{ff0000}
\definecolor{color4}{HTML}{00ff00}
\newcommand\Dganttbar[5]{
      \ganttbar[#5]{#1}{#3}{#4}\ganttbar[inline,bar label font=\tiny\bfseries, #5]{#2}{#3}{#4}
}
\begin{ganttchart}[
    expand chart=\textwidth,
    hgrid,
    vgrid,
    time slot format=isodate,
    time slot unit=month
]{2017-08-01}{2024-06-01}
    \gantttitlecalendar{year} \\
    \Dganttbar{(c)}{$j$}{2018-01-01}{2018-06-01}{bar/.append style={fill=color1}}
    \Dganttbar{(c)}{$j$}{2019-02-01}{2020-07-01}{bar/.append style={fill=color1}}
    % \Dganttbar{$i'$}{$j$}{2018-07-01}{2019-01-01}{bar/.append style={fill=color1}}
    \ganttvrule[]{$t$}{2017-12-01}
    \ganttvrule[
    vrule label node/.append style={anchor=north west}
    ]{\hspace{-2pt} $\mathcal{T}_{jt}^{MT}$}{2017-12-01}
    \ganttvrule[]{}{2018-06-01}
    \ganttvrule[]{$t'$}{2019-01-01}
    \ganttvrule[
    vrule label node/.append style={anchor=north west}
    ]{\hspace{-2pt} $\mathcal{T}_{jt'}^{MT}$}{2019-01-01}
    \ganttvrule[]{}{2019-07-01}
\end{ganttchart}

        % \end{tikzpicture}
        \caption{Examples showing the maintenance-related time-parametric sets for aircraft $i$. (a) $\mathcal{T}_{t'}^{s}$ refers to the previous $M$ periods to period $t'$. $\mathcal{T}_{t}^{m}$ refers to the periods where a check cannot be planned after starting a check in period $t$. Finally, $\mathcal{T}_{t}^{M}$ refers to the periods where a check needs to be scheduled, after starting a check in period $t$. (b) The maintenance assignment for aicraft $i$ is $m_{it_1t_2}=1$. $\mathcal{T}_A \cup \mathcal{T}_B$ represents the set of periods $t$ for which $(t_1, t_2) \in \mathcal{TTT}_t$. Also, $(j, t, t') \in \mathcal{J}\mathcal{T}\mathcal{T}_{it_1t_2}$ since the assignment is bounded by $t_1 + M$ and $t_2$. (c) $Rct_{i}^{Init}$ is equal to 55, meaning a check has to be planned between months number 25 and 54. The maintenance assignment for aircraft $i$ is $m_{it_1t_2}=1$ and, since it includes the last period, it only represents one operation. (d) $\mathcal{T}_{jt}^{MT}$ refers to the periods where the assignment of mission $j$ needs to be kept. In this case, the size is 6. The mission assignments for aircraft $i$ to mission $j$ are $a_{ijt_1t_2}=1$ and $a_{ijt'_1t'_2}=1$ and $(t'_1, t'_2) \in \mathcal{T}\mathcal{T}\mathcal{J}_{jt''}$.}
        \label{fig:gantt_windows}
    \end{figure}

    \subsubsection{Consumption derived parameters}\label{subsubsec:consumption}

    The following parameters are derived from the basic input parameters. They are used in constraints (\ref{eq:cycle_hours1}) - (\ref{eq:cycle_hours3}). They are transformations of the default and mission flight hours consumption under different time circumstances.

    \begin{align}
        & U^{\prime}_{t_1t_2} = U^{min} (t_2 - t_1+1) \notag \\
        % & U^{\prime\prime}_{t_1t_2} = U^{min} \sum_{t=t_1-1}^{t_2} (t_2 - t) = U^{min} (t_2 - t_1+1) (t_2 - t_1) \div 2  \notag \\
        & H^\prime_{jt_1t_2} = (H_j - U^{min}) (t_2 - t_1+1) \notag \\
        & H^{\prime\prime}_{jt_1t_2t'} = (H_j- U^{min}) \sum_{t=t_1-1}^{t_2} (t' - t) = (H_j- U^{min}) (t_2 - t_1 + 1) (2 t' - t_1 - t_2) \div 2 \notag
    \end{align}


\subsection{Variables}

    The following decision variables control the assignment of missions and checks to aircraft.

    \begin{tabular}{p{8mm}p{147mm}}
        $a_{ijtt'}$ &  =1 if aircraft $i$ starts a new assignment to mission $j$ at the beginning of period $t$ and finishes it at the end of period $t'$. \\  
        $m_{itt'}$  &  =1 if aircraft $i \in I$ starts a check at the beginning of period $t \in \mathcal{T}^{M_{Init}}_i$ and then starts the next one at the beginning of period $t' \in \mathcal{T}^M_{t}$, 0 otherwise. \\
    \end{tabular}

    Auxiliary variables control the status of each aircraft or group of aircraft.

    \begin{tabular}{p{8mm}p{147mm}}
        $rft_{it}$  &  remaining flight time (continuous) for aircraft $i \in I$ at the end of period $t \in \mathcal{T}$. \\  
        $e^{A}_{kts}$ & Continuous, positive. Elastic variable for serviceability constraint (\ref{eq:serviceability-cluster}). \\
        $e^{H}_{kts}$ & Continuous, positive. Elastic variable for sustainability constraint (\ref{eq:sustainability-cluster}). \\
        $e^{C}_{ts}$ & Continuous, positive. Elastic variable for capacity constraint (\ref{eq:capacity1}). \\
    \end{tabular}

    \paragraph{Fixed values}

    Note that $a_{ijtt'}$ and $m_{itt'}$ are initially set up to 0 for all aircraft already in maintenance at the beginning of the planning horizon for the remaining time periods of the check. $N_{t}$ is calculated based on this information.
Similarly, for aircraft that have not yet complied with their minimum mission assignment duration at the beginning of the planning horizon, $a_{ijtt'}$ is fixed to comply with the constraints.

\subsection{Objective function and constraints}

    Objective (\ref{eq:objective1}) minimizes the deviations number of checks while at the same time maximizing the monthly total flight hours potential of the fleet.

    \begin{align}
        & \text{Min}\; 
        \sum_{\substack{
                k \in \mathcal{K}, \\ t \in \mathcal{T}, \\ s \in \mathcal{S}
                }
            } PA_s e^{A}_{kts} 
        + \sum_{\substack{
                k \in \mathcal{K}, \\ t \in \mathcal{T}, \\ s \in \mathcal{S}
                } 
            } PH_s e^{H}_{kts}
        + \sum_{t \in \mathcal{T}, s \in \mathcal{S}} PC_s e^{C}_{ts}
        \label{eq:objective1}
    \end{align}

    The following constraints are used in the model:
    \begin{align}
        % maximum capacity1:
        & \sum_{\substack{i \in \mathcal{I}, \\ (t_1, t_2) \in \mathcal{T}\mathcal{T}\mathcal{T}_{t}}} m_{it_1t_2} + N_t \leq C^{max} + \sum_{s \in \mathcal{S}} e^{C}_{ts}
                & t \in \mathcal{T} \label{eq:capacity1}\\
        % min assignments:
        & \sum_{\substack{i \in \mathcal{IJ}_j, \\ (t_1, t_2) \in \mathcal{T}\mathcal{T}\mathcal{J}_{jt}}} a_{ijt_1t_2} \geq R_j
                & j \in \mathcal{J}, t \in \mathcal{TJ}_j  \label{eq:missionres}\\
        % just doing one thing at any given time:
        & \sum_{\substack{(t_1, t_2) \in \\ \mathcal{T}\mathcal{T}\mathcal{T}_{t}}} m_{it_1t_2} + \sum_{\substack{j \in \\ \mathcal{JT}_t \cap \mathcal{JI}_i}} \sum_{\substack{(t_1, t_2) \in \\ \mathcal{T}\mathcal{T}\mathcal{J}_{jt}}} a_{ijt_1t_2} \leq 1 
                & t \in \mathcal{T}, i \in \mathcal{I} \label{eq:state}
    \end{align}

    Maintenance capacity is controlled by (\ref{eq:capacity1}). The aircraft requirements of missions are defined by (\ref{eq:missionres}). Constraints (\ref{eq:state}) ensure that an aircraft can only be used for one mission or undergo check in the same period.

    \begin{align}
        & \sum_{\substack{(j, t, t') \in \\ \mathcal{J}\mathcal{T}\mathcal{T}_{i1t_1}}} a_{ijtt'} H^\prime_{jtt'} + U^{\prime}_{1t_1} \leq Rft^{Init}_i + H^{max} (1 - m_{it_1t_2}) 
                & i \in \mathcal{I}, t_1 \in \mathcal{T}^{M_{Init}}_i, t_2 \in \mathcal{T}^M_{t_1} \label{eq:cycle_hours1}\\
        & \sum_{\substack{(j, t, t') \in \\ \mathcal{J}\mathcal{T}\mathcal{T}_{it_1t_2}}} a_{ijtt'} H^\prime_{jtt'} + U^{\prime}_{t_1t_2} \leq H^{max} + H^{max} (1 - m_{it_1t_2}) 
                & i \in \mathcal{I}, t_1 \in \mathcal{T}^{M_{Init}}_i, t_2 \in \mathcal{T}^M_{t_1} \label{eq:cycle_hours2}\\
        & \sum_{\substack{(j, t, t') \in \\ \mathcal{J}\mathcal{T}\mathcal{T}_{it_2|\mathcal{T}|}}} a_{ijtt'} H^\prime_{jtt'} + U^{\prime}_{t_2|\mathcal{T}|} \leq H^{max} + H^{max} (1 - m_{it_1t_2}) 
                & i \in \mathcal{I}, t_1 \in \mathcal{T}^{M_{Init}}_i, t_2 \in \mathcal{T}^M_{t_1} \label{eq:cycle_hours3}
    \end{align}

        Constraints (\ref{eq:cycle_hours1}) - (\ref{eq:cycle_hours3}) limit the total flight hours of any given aircraft between checks. Constraints (\ref{eq:cycle_hours1}) takes into account periods since the beginning of the planning horizon until the first check, \ref{eq:cycle_hours2} takes into account periods between the first and second check and \ref{eq:cycle_hours3} takes into account periods between the second check and the end of the planning horizon.

    \begin{align}
       & \sum_{\substack{i \in \mathcal{IK}_k, \\ (t_1, t_2) \in \mathcal{T}\mathcal{T}\mathcal{T}_{t}}} m_{it_1t_2} + N^{Clust}_{kt} \leq A^{Clust}_{kt} + \sum_{s \in \mathcal{S}} e^{A}_{kts}
            & k \in \mathcal{K}, t \in \mathcal{T} \label{eq:serviceability-cluster}\\
       & \sum_{i \in \mathcal{IK}_k} rft_{it} \geq H^{Clust}_{kt} + \sum_{s \in \mathcal{S}} e^{H}_{kts}
            & k \in \mathcal{K}, t \in \mathcal{T} \label{eq:sustainability-cluster}
    \end{align}

    Constraints (\ref{eq:serviceability-cluster}) guarantee a minimum serviceability of aircraft for each cluster $k$. A cluster is defined by the largest group of aircraft that is required exclusively for at least one mission. 
    Constraints (\ref{eq:sustainability-cluster}) ensure there is a minimum amount of remaining flight hours for each cluster $k$.

    \begin{align}
        % remaining flight time
         & rft_{it} \leq rft_{i(t-1)} + H^M \sum_{\substack{(t_1, t_2) \in \\ \mathcal{T}\mathcal{T}\mathcal{T}_{t}}} m_{it_1t_2} - U^{min} - \sum_{\substack{j \in \mathcal{J}_t \cap \mathcal{O}_i, \\ (t_1, t_2) \in \mathcal{T}\mathcal{T}\mathcal{J}_{jt}}} a_{ijt_1t_2} (H_j - U^{min})
            & t =1, ..., \mathcal{T}, i \in \mathcal{I} \label{eq:rft_upper}\\
        & rft_{i0} = Rft^{Init}_i
               & i \in \mathcal{I} \label{eq:rft_initial}\\
        & rft_{it} \geq H^M \sum_{\substack{(t_1, t_2) \in \\ \mathcal{T}\mathcal{T}\mathcal{T}_{t}}} m_{it_1t_2}
                & t \in \mathcal{T}, i \in \mathcal{I}\label{eq:rft_lower}\\ 
        & rft_{it} \in [0,H^M]
                & t \in \mathcal{T}, i \in \mathcal{I} \label{eq:mu}
    \end{align}

    Constraints (\ref{eq:rft_upper}) - (\ref{eq:mu}) define the remaining flight time in each aircraft based on the planned missions and maintenances.

    \begin{align}
       % For now, we only want one or possibly two maintenances...
        & \sum_{\substack{t_1 \in \mathcal{T}^{M_{Init}}_i, \\ t_2 \in \mathcal{T}^M_{t_1} \cup |\mathcal{T}|}} m_{it_1t_2} =  1 
          & i \in \mathcal{I}\label{eq:num_maint} \\
        & m_{it_1t_2} =  0
          & t_1 \notin \mathcal{T}^{M_{Init}}_i, t_2 \in \mathcal{T}, i \in \mathcal{I} \label{eq:maint_forbid} \\
        & m_{it_1t_2} =  0
          & t_1 \in \mathcal{T}^{M_{Init}}_i, t_2 \notin \mathcal{T}^M_{t_1} \cup |\mathcal{T}|, i \in \mathcal{I} \label{eq:maint_forbid2}
    \end{align}

    Constraints (\ref{eq:num_maint}) limit the number of maintenance assignments per aircraft to one. This implies that each aircraft will get strictly one or two maintenances over the whole planning horizon.

    Constraints (\ref{eq:maint_forbid}) - (\ref{eq:maint_forbid2}) limit the combinations of maintenance schedules that can be done during the horizon assuming that the first maintenance is dependant in the initial state of the aircraft and the second maintenance (if any) is limited by the moment at which the first maintenance was planned.

\end{document}