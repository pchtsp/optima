\documentclass[a4paper,onecolumn,fleqn]{article}
\textwidth 17cm \textheight 247mm
\topmargin -4mm
\hoffset -9mm \voffset -14mm
%\setlength{\topmargin}{-0.8truecm}
\setlength{\oddsidemargin}{0.3cm}
\setlength{\evensidemargin}{0.3cm}
\setlength{\columnsep}{8mm}
\setlength{\parindent}{0mm}
\setlength{\parskip}{-2.0ex}
\setlength{\mathindent}{0mm}
\flushbottom


\usepackage{epsfig}
%\usepackage{timesnew}
% \usepackage{harvard}
% \usepackage[]{natbib}
\usepackage{amsmath}
\usepackage{booktabs}

\usepackage[latin1]{inputenc}
% \usepackage{aeguill}
\usepackage[english]{babel}
\usepackage{caption}
\usepackage{pgfgantt}


\setlength{\parskip}{2ex} \pagestyle{myheadings}
% \markright{\hspace*{3.8cm} \textit{MOSIM18 - June 27-29, 2018 - Toulouse - France}}

\renewcommand{\thepage}{}
\renewcommand{\refname}{REFERENCES}



\makeatletter
\renewcommand\section{\@startsection{section}{1}{\z@}%
                       {-6\p@ \@plus -0\p@ \@minus -0\p@}%
                       {2\p@ \@plus 0\p@ \@minus 0\p@}%
                       {\normalsize\textbf}}

\renewcommand\subsection{\@startsection{subsection}{2}{\z@}%
                       {-6\p@ \@plus -0\p@ \@minus -0\p@}%
                       {2\p@ \@plus 0\p@ \@minus 0\p@}%
                       {\normalsize\textbf}}

\renewcommand\subsubsection{\@startsection{subsubsection}{3}{\z@}%
                       {-6\p@ \@plus -0\p@ \@minus -0\p@}%
                       {1\p@ \@plus 0\p@ \@minus 0\p@}%
                       {\normalsize\itshape\bfseries}}
\makeatother


\begin{document}

\section{Mathematical formulation}
  \label{sec:model}

The following model provides a tight MIP formulation that solves the Military Flight and Maintenance Problem. Maintenance operations will be referred as "checks".

\subsection{Input data}
    \subsubsection{Basic sets}

        \begin{tabular}{p{15mm}p{140mm}}
            $i \in \mathcal{I}$     &  aircraft. \\
            $t \in \mathcal{T}$     &  time periods included in the planning horizon. \\
            $j \in \mathcal{J}$     &  missions. \\
            $k \in \mathcal{K}$     &  cluster of aircraft that share the same functionality. \\
        \end{tabular}

    \subsubsection{Mission parameters}

        \begin{tabular}{p{15mm}p{125mm}p{15mm}}
            $H_j$             & amount of flight time required by mission $j$. & hours. \\
            $R_j$             & number of aircraft required by mission $j$. & aircraft. \\
            $MT_j$            & minimum number of consecutive periods an aircraft has to be assigned to mission $j$. & periods. \\
            $U^{min}$         & default aircraft flight time if it is not assigned to any mission nor in maintenance.& hours \\
            $M$               & check duration in number of periods. & periods. \\
        \end{tabular}

    \subsubsection{Maintenance parameters}

        \begin{tabular}{p{15mm}p{125mm}p{15mm}}
            $C^{max}$         & maximum number of simultaneous checks. & aircraft. \\
            $E^{max}$         & maximum number of periods between two consecutive checks. & periods. \\
            $E^{min}$         & minimum number of periods between two consecutive checks. & periods. \\
            $H^{max}$         & remaining flight time after a check. & hours. \\
        \end{tabular}

    \subsubsection{Fleet parameters}

        \begin{tabular}{p{15mm}p{125mm}p{15mm}}
            $N_t$               & number of aircraft known to be in maintenance in period $t$. & aircraft. \\
            $N^{Clust}_{kt}$    & number of aircraft in cluster $k$ known to be in maintenance in period $t$. & aircraft. \\
            $A^{Clust}_{kt}$    & maximum number of aircraft in cluster $k$ that can be simultaneously in maintenance in period $t$. & aircraft. \\
            $H^{Clust}_{kt}$    & minimum number of total remaining flight time for cluster $k$ at period $t$. & hours. \\
            $Rft^{Init}_i$      & remaining flight time for aircraft $i$ at the start of the planning horizon.  & hours.  \\
            $Rct^{Init}_i$      & remaining calendar time for aircraft $i$ at the start of the planning horizon.  & periods. \\
        \end{tabular}

    \subsubsection{Parametric sets}

        \begin{tabular}{p{15mm}p{140mm}}
            $t \in \mathcal{TJ}_j$     &  time periods $t \in \mathcal{T}$ in which mission $j$ is active. \\
            $j \in \mathcal{JT}_t$    &  missions $j \in \mathcal{J}$ to be realized in period $t$. \\
            $i \in \mathcal{IJ}_j$     &  aircraft $i \in \mathcal{I}$ that can be assigned to mission $j$. \\
            $i \in \mathcal{IK}_k$     &  aircraft $i \in \mathcal{I}$ that are included in cluster $k$. One aircraft can belong to more than one cluster. \\
            $j \in \mathcal{JI}_i$     &  missions $j \in \mathcal{J}$ for which aircraft $i$ can be used. \\
            $i \in \mathcal{A}^{Init}_j$  & aircraft $i \in \mathcal{I}$ that have mission $j$ pre-assigned in the previous period to the start of the planning horizon. \\
        \end{tabular}

    \subsubsection{Time-related parametric sets}

        Several intermediate sets have been defined based on the input data in order to simplify constraint formulation.

        \begin{tabular}{p{30mm}p{125mm}}
            $t \in \mathcal{T}^s_t$ &  time periods $t' \in \mathcal{T}$ such that $t' \in \{ \max{\{1, t - M+1\}},  ..., {t}\}$ (figure \ref{fig:gantt_windows}a). \\
            % $\mathcal{T}^S_t$ &  time periods $t' \in \mathcal{T}$ such that $t' \in \{ {t},  ..., \min{\{|\mathcal{T}|, t + M-1\}}\}$ (figure \ref{fig:gantt_windows}a). \\
            $t' \in \mathcal{T}^m_t$ &  time periods $t' \in \mathcal{T}$ such that $t' \in \{ {t}, ..., \min{\{|\mathcal{T}|, t + M + E^{min}-1\}}\}$ (figure \ref{fig:gantt_windows}a). \\
            $t' \in \mathcal{T}^M_t$ &  time periods $t' \leq |\mathcal{T}| - E^{max} - M$ such that $t' \in \{ t + M + E^{min}-1 , ...,  t + M + E^{max}-1 \}$ (figure \ref{fig:gantt_windows}a). \\
            $t \in \mathcal{T}^{m_{Init}}_i$ &  time periods $t \in \mathcal{T}$ such that $t \in \{ 1, ..., \max{\{0, Rct^{Init}_i - E^{max} + E^{min} \}}\}$ (figure \ref{fig:gantt_windows}b). \\
            $t \in \mathcal{T}^{M_{Init}}_i$ &  time periods $t \in \mathcal{T}$ such that $t \in \{ \max{\{0, Rct^{Init}_i - E^{max} + E^{min} \}} , ...,  Rct^{Init}_i \}$ (figure \ref{fig:gantt_windows}b). \\
            $t' \in \mathcal{T}^{MT}_{jt}$ &  time periods $t' \in \mathcal{T}$ such that $t' \in \{ \max{\{1, t - MT_j\}},  ..., {t}\}$ (figure \ref{fig:gantt_windows}c). \\
            $(t_1, t_2) \in \mathcal{T}\mathcal{T}\mathcal{J}_{jt}$ & pairs of time periods $t_1, t_2 \in \mathcal{TJ}_j$ such that $t_2 \ge t_1 + MT_j-1$ and $t \in \{t_1, ..., t_2\}$. \\
            $(t_1, t_2) \in \mathcal{T}\mathcal{T}\mathcal{T}_{t}$ & pairs of time periods $t_1 \in \mathcal{T}, t_2 \in \mathcal{T}^M_{t_1}$ such that $t_1 \in \mathcal{T}^s_{t} \lor t_2 \in \mathcal{T}^s_{t}$. \\
            $(j, t, t') \in \mathcal{J}\mathcal{T}\mathcal{T}_{it_1t_2}$ & triplet composed of mission $j \in \mathcal{JI}_i$ and $t, t' \in \mathcal{TJ}_j$ such that $t' \ge t + MT_j-1$ and $t \ge t_1 + M$ and $t' < t_2$ \\
        \end{tabular}

    \begin{figure}
        \centering
        % \includegraphics[width=\linewidth]{img/gantt_timewindows_all.png}
        \begin{tikzpicture}
            % Same month implies one month
% so i need to take one month after summing the amount of months of the duration

\definecolor{color1}{HTML}{FFFFCC}
\definecolor{color2}{HTML}{FFEDA0}
\definecolor{color3}{HTML}{ff0000}
\definecolor{color4}{HTML}{00ff00}
\newcommand\Dganttbar[5]{
    \ganttbar[#5]{#1}{#3}{#4}\ganttbar[inline,bar label font=\tiny\bfseries, #5]{#2}{#3}{#4}
}
\begin{ganttchart}[
    expand chart=\textwidth,
    hgrid,
    vgrid,
    time slot format=isodate,
    time slot unit=month
]{2017-08-01}{2024-06-01}
    \gantttitlecalendar{year} \\
    \Dganttbar{(a)}{M}{2018-03-01}{2018-08-01}{bar/.append style={fill=color1}}
    \Dganttbar{(a)}{M}{2023-07-01}{2023-12-01}{bar/.append style={fill=color1}}
    \ganttvrule[]{$t$}{2018-02-01}
    \ganttvrule[]{$t'$}{2018-08-01}
    \ganttvrule[]{\hspace{25pt} $\mathcal{T}_{t'}^s$}{2018-02-01}
    \ganttvrule[
    vrule label node/.append style={anchor=north west}
    ]{\hspace{50pt} $\mathcal{T}_t^m$}{2018-08-01}
    \ganttvrule[
    vrule label node/.append style={anchor=north west}
    ]{\hspace{50pt} $\mathcal{T}_t^M$}{2021-02-01}
    \ganttvrule[]{}{2023-08-01}
\end{ganttchart}

        \end{tikzpicture}
        \begin{tikzpicture}
            % Same month implies one month
% so i need to take one month after summing the amount of months of the duration

\definecolor{color1}{HTML}{FFFFCC}
\definecolor{color2}{HTML}{FFEDA0}
\definecolor{color3}{HTML}{ff0000}
\definecolor{color4}{HTML}{00ff00}
\newcommand\Dganttbar[5]{
    \ganttbar[#5]{#1}{#3}{#4}\ganttbar[inline,bar label font=\tiny\bfseries, #5]{#2}{#3}{#4}
}
\begin{ganttchart}[
    expand chart=\textwidth,
    hgrid,
    vgrid,
    time slot format=isodate,
    time slot unit=month
]{2017-08-01}{2024-06-01}
    % \gantttitlecalendar{year} \\
    \Dganttbar{(b)}{M}{2018-03-01}{2018-08-01}{bar/.append style={fill=color1}}
    \Dganttbar{(b)}{M}{2023-07-01}{2023-12-01}{bar/.append style={fill=color1}}
    \ganttvrule[]{$t_1$}{2018-02-01}
    \ganttvrule[]{$t_2$}{2023-06-01}
    \ganttvrule[]{\hspace{-30pt} $\mathcal{T}_A$}{2018-08-01}
    \ganttvrule[]{\hspace{-30pt}$\mathcal{T}_B$}{2023-12-01}
    % \ganttvrule[
    % vrule label node/.append style={anchor=north west}
    % ]{\hspace{50pt} $\mathcal{T}_t^M$}{2021-02-01}
    % \ganttvrule[]{}{2023-08-01}
\end{ganttchart}

        \end{tikzpicture}
        \begin{tikzpicture}
            \definecolor{color1}{HTML}{FFFFCC}
\definecolor{color2}{HTML}{FFEDA0}
\definecolor{color3}{HTML}{ff0000}
\definecolor{color4}{HTML}{00ff00}
\newcommand\Dganttbar[5]{
    \ganttbar[#5]{#1}{#3}{#4}\ganttbar[inline,bar label font=\tiny\bfseries, #5]{#2}{#3}{#4}
}
\begin{ganttchart}[
    expand chart=\textwidth,
    hgrid,
    vgrid,
    time slot format=isodate,
    time slot unit=month
]{2017-08-01}{2024-06-01}
    % \gantttitlecalendar{year} \\
    \Dganttbar{(c)}{M}{2020-10-01}{2021-03-01}{bar/.append style={fill=color1}}
    % \Dganttbar{(b)}{M}{2020-10-01}{2021-03-01}{bar/.append style={fill=color1}}
    \ganttvrule[]{}{2019-08-01}
    \ganttvrule[]{\hspace{-120pt}$\mathcal{T}^{M_{Init}}_i$}{2021-04-01}
    \ganttvrule[
        vrule/.style={dotted, line width=0.7pt}
    ]{$t_1$}{2020-09-01}
    \ganttvrule[
        vrule/.style={dotted, line width=0.7pt}
    ]{$t_2$}{2024-06-01}
    % {draw=black, fill=green!50}
    % dashed, line width=1pt, very thick, blue
    % \ganttvrule[
    % vrule label node/.append style={anchor=north west}
    % ]{}{2020-02-01}
\end{ganttchart}

        \end{tikzpicture}
        \begin{tikzpicture}
            \definecolor{color1}{HTML}{FFFFCC}
\definecolor{color2}{HTML}{FFEDA0}
\definecolor{color3}{HTML}{ff0000}
\definecolor{color4}{HTML}{00ff00}
\newcommand\Dganttbar[5]{
      \ganttbar[#5]{#1}{#3}{#4}\ganttbar[inline,bar label font=\tiny\bfseries, #5]{#2}{#3}{#4}
}
\begin{ganttchart}[
    expand chart=\textwidth,
    hgrid,
    vgrid,
    time slot format=isodate,
    time slot unit=month
]{2017-08-01}{2024-06-01}
    \gantttitlecalendar{year} \\
    \Dganttbar{(c)}{$j$}{2018-01-01}{2018-06-01}{bar/.append style={fill=color1}}
    \Dganttbar{(c)}{$j$}{2019-02-01}{2020-07-01}{bar/.append style={fill=color1}}
    % \Dganttbar{$i'$}{$j$}{2018-07-01}{2019-01-01}{bar/.append style={fill=color1}}
    \ganttvrule[]{$t$}{2017-12-01}
    \ganttvrule[
    vrule label node/.append style={anchor=north west}
    ]{\hspace{-2pt} $\mathcal{T}_{jt}^{MT}$}{2017-12-01}
    \ganttvrule[]{}{2018-06-01}
    \ganttvrule[]{$t'$}{2019-01-01}
    \ganttvrule[
    vrule label node/.append style={anchor=north west}
    ]{\hspace{-2pt} $\mathcal{T}_{jt'}^{MT}$}{2019-01-01}
    \ganttvrule[]{}{2019-07-01}
\end{ganttchart}

        \end{tikzpicture}
        \caption{Examples showing the maintenance-related time-parametric sets for aircraft $i$. (a) $\mathcal{T}_{t'}^{s}$ refers to the previous $M$ periods to period $t'$. $\mathcal{T}_{t}^{m}$ refers to the periods where a check cannot be planned after starting a check in period $t$. Finally, $\mathcal{T}_{t}^{M}$ refers to the periods where a check needs to be scheduled, after starting a check in period $t$. (b) The maintenance assignment for aicraft $i$ is $m_{it_1t_2}=1$. $\mathcal{T}_A \cup \mathcal{T}_B$ represents the set of periods $t$ for which $(t_1, t_2) \in \mathcal{TTT}_t$. (c) $Rct_{i}^{Init}$ is equal to 55, meaning a check has to be planned between months number 25 and 54. The maintenance assignment for aircraft $i$ is $m_{it_1t_2}=1$ and, since it includes the last period, it only represents one operation. (d) $\mathcal{T}_{jt}^{MT}$ refers to the periods where the assignment of mission $j$ needs to be kept. In this case, the size is 6. The mission assignments for aircraft $i$ to mission $j$ are $a_{ijt_1t_2}=1$ and $a_{ijt'_1t'_2}=1$}
        \label{fig:gantt_windows}
    \end{figure}

    \subsubsection{Consumption derived parameters}\label{subsubsec:consumption}

    The following parameters are derived from the basic input parameters. They are used in constraints (\ref{eq:cycle_hours1}) - (\ref{eq:cycle_rem_hours3}). They are transformations of the default and mission flight hours consumption under different time circumstances.

    \begin{align}
        & U^{min}_{t_1t_2} = U^{min} \sum_{t=t_1}^{t_2} (t_2 - t) \notag \\
        & H'_{jt_1t_2} = (H_j - U^{min}) (t_2 - t_1) \notag \\
        & H''_{jt_1t_2t'} = \sum_{t=t_1}^{t_2} (H_j- U^{min}) (t' - t) \notag
    \end{align}


\subsection{Variables}

    The following decision variables control the assignment of missions and checks to aircraft.

    \begin{tabular}{p{8mm}p{147mm}}
        $a_{ijtt'}$ &  =1 if aircraft $i$ starts a new assignment to mission $j$ at the beginning of period $t$ and finishes it at the end of period $t'$. \\  
        $m_{itt'}$  &  =1 if aircraft $i \in I$ starts a check at the beginning of period $t \in \mathcal{T}$ and then starts the next one at the beginning of period $t'$, 0 otherwise. \\
        $h_{itt'}$  &  Continuous. Sum of the monthly remaining flight hours for aircraft $i \in I$ for periods between the beginning of period $t$ and the beginning of period $t'$. \\
    \end{tabular}

    \paragraph{Fixed values}

    Note that $a_{ijtt'}$ and $m_{itt'}$ are initially set up to 0 for all aircraft already in maintenance at the beginning of the planning horizon for the remaining time periods of the check. $N_{t}$ is calculated based on this information.
Similarly, for aircraft that have not yet complied with their minimum mission assignment duration at the beginning of the planning horizon, $a_{ijtt'}$ is fixed to comply with the constraints.

\subsection{Objective function and constraints}

    Objective (\ref{eq:objective1}) minimizes the number of checks while at the same time maximizing the monthly total flight hours potential of the fleet.

    \begin{align}
        & \text{Min}\; \sum_{\substack{i \in \mathcal{I}, \\ t_1 \in \mathcal{T}^{M_{Init}}_i, \\ t_2 \in \mathcal{T}^M_{t_1} - |\mathcal{T}|}} m_{it_1t_2} \times H^{max} \times (t_2 - t_1) - h_{it_1t_2}  \label{eq:objective1}
    \end{align}
    
    The first term counts all the remaining flight hours given to aircraft following checks and the second term quantifies the amount of remaining flight hours for all aircraft at each period. These two objectives have the same units, can be easily compared and ensure the aircraft are used in the most efficient way.

    The following constraints are used in the model:
    \begin{align}
        % maximum capacity1:
        & \sum_{\substack{i \in \mathcal{I}, \\ (t_1, t_2) \in \mathcal{T}\mathcal{T}\mathcal{T}_{t}}} m_{it_1t_2} + N_t \leq C^{max}
          & t \in \mathcal{T} \label{eq:capacity1}\\
        % min assignments:
        & \sum_{\substack{i \in \mathcal{IJ}_j, \\ (t_1, t_2) \in \mathcal{T}\mathcal{T}\mathcal{J}_{jt}}} a_{ijt_1t_2} \geq R_j
                & j \in \mathcal{J}, t \in \mathcal{TJ}_j  \label{eq:missionres}\\
        % just doing one thing at any given time:
        & \sum_{\substack{(t_1, t_2) \in \\ \mathcal{T}\mathcal{T}\mathcal{T}_{t}}} m_{it_1t_2} + \sum_{\substack{j \in \\ \mathcal{JT}_t \cap \mathcal{JI}_i}} \sum_{\substack{(t_1, t_2) \in \\ \mathcal{T}\mathcal{T}\mathcal{J}_{jt}}} a_{ijt_1t_2} \leq 1 
                & t \in \mathcal{T}, i \in \mathcal{I} \label{eq:state}
    \end{align}

    Maintenance capacity is controlled by (\ref{eq:capacity1}). The aircraft requirements of missions are defined by (\ref{eq:missionres}). Constraints (\ref{eq:state}) ensure that an aircraft can only be used for one mission or undergo check in the same period.

    \begin{align}
       & \sum_{\substack{i \in \mathcal{IK}_k, \\ (t_1, t_2) \in \mathcal{T}\mathcal{T}\mathcal{T}_{t}}} m_{it_1t_2} + N^{Clust}_{kt} \leq A^{Clust}_{kt}
        &k \in \mathcal{K}, t \in \mathcal{T} \label{eq:serviceability-cluster}
       % & \sum_{i \in \mathcal{IK}_k} rft_{it} \geq H^{Clust}_{kt}
       %  &k \in \mathcal{K}, t \in \mathcal{T} \label{eq:sustainability-cluster}
    \end{align}

    Constraints (\ref{eq:serviceability-cluster}) guarantee a minimum serviceability of aircraft for each cluster $k$. A cluster is defined by the largest group of aircraft that is required exclusively for at least one mission. 
    % Constraints (\ref{eq:sustainability-cluster}) ensure there is a minimum amount of remaining flight time for each cluster $k$.

    \begin{align}
        & \sum_{\substack{(j, t, t') \in \\ \mathcal{J}\mathcal{T}\mathcal{T}_{i1t_1}}} a_{ijtt'} H'_{jtt'} + U^{min}_{1t_1} \leq Rft^{Init}_i + H^{max} (1 - m_{it_1t_2}) 
                & t_1 \in \mathcal{T}, t_2 \in \mathcal{T}^M_{t_1}, i \in \mathcal{I}\label{eq:cycle_hours1}\\
        & \sum_{\substack{(j, t, t') \in \\ \mathcal{J}\mathcal{T}\mathcal{T}_{it_1t_2}}} a_{ijtt'} H'_{jtt'} + U^{min}_{t_1t_2} \leq H^{max} + H^{max} (1 - m_{it_1t_2}) 
                & t_1 \in \mathcal{T}, t_2 \in \mathcal{T}^M_{t_1}, i \in \mathcal{I}\label{eq:cycle_hours2}\\
        & \sum_{\substack{(j, t, t') \in \\ \mathcal{J}\mathcal{T}\mathcal{T}_{it_2|\mathcal{T}|}}} a_{ijtt'} H'_{jtt'} + U^{min}_{t_2|\mathcal{T}|} \leq H^{max} + H^{max} (1 - m_{it_1t_2}) 
                & t_1 \in \mathcal{T}, t_2 \in \mathcal{T}^M_{t_1}, i \in \mathcal{I}\label{eq:cycle_hours3}
    \end{align}

        Constraints (\ref{eq:cycle_hours1}) - (\ref{eq:cycle_hours3}) limit the total flight hours of any given aircraft between checks. Constraints (\ref{eq:cycle_hours1}) takes into account periods since the beginning of the planning horizon until the first check, \ref{eq:cycle_hours2} takes into account periods between the first and second check and \ref{eq:cycle_hours3} takes into account periods between the second check and the end of the planning horizon.

    \begin{align}
        & h_{i1t_1} \geq (t_1 - 1) Rft^{Init}_i m_{it_1t_2} - \sum_{\substack{(j, t, t') \in \\ \mathcal{J}\mathcal{T}\mathcal{T}_{i1t_1}}} a_{ijtt'} H''_{jtt't_1} - U^{min}_{1t_1} 
            & t_1 \in \mathcal{T}, t_2 \in \mathcal{T}^M_{t_1}, i \in \mathcal{I}\label{eq:cycle_rem_hours1}\\
        & h_{it_1t_2} \geq (t_2 - t_1) H^{max} m_{it_1t_2} - \sum_{\substack{(j, t, t') \in \\ \mathcal{J}\mathcal{T}\mathcal{T}_{it_1t_2}}} a_{ijtt'} H''_{jtt't_2} - U^{min}_{t_1t_2}
            &t_1 \in \mathcal{T}, t_2 \in \mathcal{T}^M_{t_1}, i \in \mathcal{I}\label{eq:cycle_rem_hours2}\\
        & h_{it_2|\mathcal{T}|} \geq (|\mathcal{T}|- t_2) H^{max} m_{it_1t_2} - \sum_{\substack{(j, t, t') \in \\ \mathcal{J}\mathcal{T}\mathcal{T}_{it_2|\mathcal{T}|}}} a_{ijtt'} H''_{jtt'|\mathcal{T}|} - U^{min}_{t_2|\mathcal{T}|}
            & t_1 \in \mathcal{T}, t_2 \in \mathcal{T}^M_{t_1}, i \in \mathcal{I}\label{eq:cycle_rem_hours3}
    \end{align} 

    Constraints (\ref{eq:cycle_rem_hours1}) - (\ref{eq:cycle_rem_hours3}) define the sum of remaining flight hours each aircraft has in between checks. Similarly to (\ref{eq:cycle_hours1}) - (\ref{eq:cycle_hours3}), their difference lies in the three possible intervals produced by assigning at most two checks. The derived parameters used in these constraints can be found in section \ref{subsubsec:consumption}.

    \begin{align}
       % For now, we only want one or possibly two maintenances...
        & \sum_{\substack{t_1 \in \mathcal{T}^{M_{Init}}_i, \\ t_2 \in \mathcal{T}^M_{t_1} \cup |\mathcal{T}|}} m_{it_1t_2} =  1 
          & i \in \mathcal{I}\label{eq:num_maint} \\
        % & m_{it'} + m_{it} \leq 1
        %   & t \in \mathcal{T}, t' \in \mathcal{T}^m_t, i \in \mathcal{I}\label{eq:rct_min}\\  
        % & \sum_{t' \in \mathcal{T}^M_t} m_{it'} \geq  m_{it}
        %   & t \in \mathcal{T}, i \in \mathcal{I}\label{eq:rct_max}\\
        % & m_{it_1t_2} = 0
        %   & t_1 \in \mathcal{T}^{m_{Init}}_i, t_2 \in \mathcal{T}^M_{t_1} \cup \{|\mathcal{T}|\}, i \in \mathcal{I}\label{eq:rct_min_init} \\
        & m_{it_1t_2} =  0
          & t_1 \notin \mathcal{T}^{M_{Init}}_i, t_2 \in \mathcal{T}, i \in \mathcal{I} \label{eq:maint_forbid} \\
        & m_{it_1t_2} =  0
          & t_1 \in \mathcal{T}^{M_{Init}}_i, t_2 \notin \mathcal{T}^M_{t_1} \cup |\mathcal{T}|, i \in \mathcal{I} \label{eq:maint_forbid2}
    \end{align}

    Constraints (\ref{eq:num_maint}) limit the number of maintenance assignments per aircraft to one. This implies that each aircraft will get strictly one or two maintenances over the whole planning horizon.

    Constraints (\ref{eq:maint_forbid}) - (\ref{eq:maint_forbid2}) limit the combinations of maintenance schedules that can be done during the horizon assuming that the first maintenance is dependant in the initial state of the aircraft and the second maintenance (if any) is limited by the moment at which the first maintenance was planned.

\end{document}