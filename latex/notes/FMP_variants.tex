\documentclass[a4paper,11pt]{article}
\usepackage[utf8]{inputenc}
\usepackage{amsmath}

%opening
\title{Variants on the Flight and Maintenance Planning problem}
\author{}

\setlength{\topmargin}{-2.1cm}
\setlength{\textwidth}{15.8cm}
\setlength{\textheight}{26.3cm}
\setlength{\oddsidemargin}{-0.1cm}
\setlength{\evensidemargin}{-0.1cm}

\def\II{{\mathcal{I}}}
\def\TT{{\mathcal{T}}}
\def\KK{{\mathcal{K}}}

\usepackage{graphicx}
\usepackage{amssymb}

\begin{document}

\maketitle

\section{Crew scheduling variant}

    \subsection{Hypothesis}

    \begin{itemize}
     \item fixed maintenance duration.
     \item fixed duration between maintenances.
     \item homogeneous fleet.
     \item objective: minimize the number of aircraft.
    \end{itemize}

    \subsection{Notations}

    \textbf{Sets}

    \begin{tabular}{ll}
    $t\in \mathcal{T} = [\![1, m]\!]$ & Planning horizon periods. \\
    $i \in \mathcal{P} = [\![1, n]\!]$ & Working pattern.\\
    \end{tabular}

    \vskip 0.3cm

    \subsection{Parameters}

    \begin{tabular}{ll}
        $D_{off}$ & Number of consecutive non-working days. \\
        $D_{on}$ & Number of consecutive working days. \\
        $\mathcal{A}_{it}$ & Pattern $i$ covers period $t$ (circulant matrix built from $D_{on}$ and $D_{off}$).\\
        $b_{t}$ & Need of resources at period $t$. \\
    \end{tabular}

    \vskip 0.3cm

    Example of A for $D_{on}=4$ and $D_{off}=2$

    $$
    \begin{bmatrix}
        1 & 0 & 0 & \dots  & A_{i1} \\
        1 & 1 & 0 & \dots  & A_{i2} \\
        1 & 1 & 1 & \dots  & A_{i3} \\
        1 & 1 & 1 & \dots  & A_{i4} \\
        0 & 1 & 1 & \dots  & A_{i5} \\
        0 & 0 & 1 & \dots  & A_{i6} \\
        1 & 0 & 0 & \dots  & A_{i6} \\
        \vdots & \vdots & \vdots & \ddots & \vdots \\
        A_{1t} & A_{2t} & A_{3t} & \dots  & A_{nm}
    \end{bmatrix}
    $$

    \subsection{Variables}

    \begin{itemize}
     \item $x_i$: number of patterns of type $i$ to be included in the solution.
    \end{itemize}

    \subsection{Model}

    \begin{align}
        & \text{Min}\; \sum_{i \in \mathcal{P}} x_i
    \end{align}

    \begin{align}
        & \sum_{i \in \mathcal{P}} A_{it}x_{i} \geq d_j & t \in \mathcal{T} \\
        & x_{i} \geq 0,\; integer \\
    \end{align}

\clearpage
%%%%%%%%%%%%%%%%%%%%%%%%%%%%%%%%%%%%%%%%%%%%%%%%%%%%%%%%%%%%%%%%%%%%%%%%%%%%%%

\section{Set covering variant}

    \subsection{Hypothesis}

    \begin{itemize}
     \item only 1 aircraft needed per month.
     \item objective: minimize the number of aircraft.
    \end{itemize}

    \subsection{Notations}

    \textbf{Sets}

    \begin{tabular}{ll}
    $t\in \mathcal{T} = [\![1, m]\!]$ & Planning horizon periods. \\
    $i \in \mathcal{P} = [\![1, n]\!]$ & Working pattern.\\
    \end{tabular}

    \vskip 0.3cm

    \subsection{Parameters}

    \begin{tabular}{ll}
        $\mathcal{A}_{it}$ & Pattern $i$ covers period $t$.\\
    \end{tabular}

    \vskip 0.3cm

    \subsection{Variables}

    \begin{itemize}
     \item $x_i$: is pattern $i$ in the solution or not.
    \end{itemize}

    \subsection{Model}

    \begin{align}
        & \text{Min}\; \sum_{i \in \mathcal{P}} x_i
    \end{align}

    \begin{align}
        & \sum_{i \in \mathcal{P}} A_{it}x_{i} \geq 1 & t \in \mathcal{T} \\
        & x_{i} \geq 0,\; binary \\
    \end{align}

\clearpage
%%%%%%%%%%%%%%%%%%%%%%%%%%%%%%%%%%%%%%%%%%%%%%%%%%%%%%%%%%%%%%%%%%%%%%%%%%%%%%

\section{Scheduling variant}

    \subsection{Hypothesis}

    \begin{itemize}
     \item durée de maintenance définie
     \item fenêtres de temps pour commencer les maintenances
     \item objectif: lisser les maintenances dans les fenêtres de temps prédéfinies
    \end{itemize}


    \subsection{Notations}

    \noindent{
    \begin{tabular}{ll}
        $t\in \TT = [\![1, T]\!]$ & Discrétisation temporelle de l'horizon de planigfication. \\
        $i \in \II = [\![1, I]\!]$ & Ensemble des avions.\\
        $k \in \KK_i = [\![1, K_i]\!]$ & Ensemble des maintenances à planifier successivement.\\ 
    \end{tabular}
    }

    \vskip 0.3cm

    \textbf{Paramètres}


    \begin{tabular}{ll}
        $\mathbf{Da}_{i,k}$ & Durée nominale de la maintenance $k$ de l'avion $i$.\\
        $\mathbf{l}_{i}$ & Durée mininale du cycle de vol de l'avion $i$.\\
        $\mathbf{{To}}_{i,k}$ & Pas de temps le plus tôt pour commencer la maintenance $k$ de l'avion $i$.\\
        $\mathbf{{Ta}}_{i,k}$ & Pas de temps le plus tard pour terminer la maintenance $k$ de l'avion $i$.\\
    \end{tabular}

    % \caption{Definition and notations the challenge ROADEF}\label{contraintesROADEF}
    % \end{table}

    \subsection{Définition des variables}

    \begin{itemize}
     \item $d_{i,k,t}\in \{0,1\}$; variables binaires pour indiquer les dates d'arrêts, "step varaibles" 
    $d_{i,k,t}=1$ si l'arrêt du cycle $k$ du réacteur nucléaire $i$ a été commencé avant la semaine $t$.
    $d_{i,k,t}-d_{i,k,t- {\mathbf{Da}_{i,k}}} \in \{0,1\}$ vaut $1$ uniquement pendant a maintenance $(i,k)$.


     \item $m,M\geqslant 0$: variables continues pour linéariser la fonction objectif
    \end{itemize}


    \subsection{Formulation PLNE}


    \begin{eqnarray}
    \displaystyle \min_{d,m,M} & M-m
     & \\
    \forall i , k, t, & d_{i,k,t-1}\leqslant d_{i,k,t} & \label{PANprecedence}\\
    \forall i , k, & d_{i,k,\mathbf{To}_{i,k}-1}\leqslant 0 & \label{PANtw0} \\
    \forall i , k, & d_{i,k,\mathbf{Ta}_{i,k}}\geqslant 1 & \label{PANtw1}\\
    \forall i , k, t, & d_{i,k,t}\leqslant d_{i,k-1,t-l_i-\mathbf{Da}_{i,k}} & \label{PANprecedence2}\\
    \forall t, & \displaystyle \sum_{i,k} d_{i,k,t}-d_{i,k,t- {\mathbf{Da}_{i,k}}} \leqslant M & \label{maxMaint}\\
    \forall t, & \displaystyle \sum_{i,k} d_{i,k,t}-d_{i,k,t- {\mathbf{Da}_{i,k}}} \geqslant m & \label{minMaint}\\
    & d \in \{0,1\}^N, m,M \geqslant 0
    \end{eqnarray}

    (\ref{PANprecedence}): contrainte induite par la définition ``step variables".

    (\ref{PANtw0}), (\ref{PANtw1}): contraintes de fenêtres de temps (écrit explicitement, ou sinon définir uniquement les variables $d_{i,k,t}$
    dans les fenêtres de temps)

    (\ref{PANprecedence2}): durée minimale d'un cycle de vol $l\geqslant 0$, contrainte de succession des cycles.

    (\ref{maxMaint}), (\ref{minMaint}): contraites pour linéariser les $\min$ et $\max$ de la focntion objectif de lissage

    \subsection{Questions ouvertes}

    \begin{itemize}
    %\item Avec $l=0$ , y a t'il un problème de référence proche ? Ordonnancement, lissage de charge ?
    \item Avec $l=0$ , y a t'il un problème de référence proche ? Ordonnancement, lissage de charge ?
    \item Est-ce un problème NP-complet? Peut on le prouver ?
     \item Cas particulier $\mathbf{Ta}_{i,k}-\mathbf{To}_{i,k}=1$? (pour lissage local)
     \item Autres fonctions objectifs?
     \item version robuste avec aléas de prolongation de durée de maintenance?
    \end{itemize}

\clearpage
%%%%%%%%%%%%%%%%%%%%%%%%%%%%%%%%%%%%%%%%%%%%%%%%%%%%%%%%%%%%%%%%%%%%%%%%%%%%%%

\section{Crew scheduling: smooth maintenances}


    \subsection{Hypothesis}

    \begin{itemize}
     \item fixed maintenance duration.
     \item range to start each maintenance.
     \item min distance between maintenances.
     \item objective: minimize the max number of maintenances.
    \end{itemize}

    \subsection{Notations}

    \textbf{Sets}

    \begin{tabular}{ll}
    $t\in \mathcal{T} = [\![1, m]\!]$ & Planning horizon periods. \\
    $i \in \mathcal{P} = [\![1, n]\!]$ & Working pattern.\\
    \end{tabular}

    \vskip 0.3cm

    \subsection{Parameters}

    \begin{tabular}{ll}
        $\mathcal{A}_{it}$ & Pattern $i$ has a maintenance in $t$.\\
    \end{tabular}

    \vskip 0.3cm

    Here, A needs to be created taking into account:
    1. The range of start of maintenances.
    2. The fixed duration of maintenances.
    3. The minimal distance between maintenances.

    \subsection{Variables}

    \begin{itemize}
     \item $x_i$: number of patterns of type $i$ to be included in the solution.
     \item $M$: maximum number of maintenance among all periods.
    \end{itemize}

    \subsection{Model}

    \begin{align}
        & \text{Min}\; M
    \end{align}

    \begin{align}
        & \sum_{i \in \mathcal{P}} A_{it}x_{i} \leq M & t \in \mathcal{T} \\
        & x_{i} \geq 0,\; integer \\
    \end{align}

\clearpage
%%%%%%%%%%%%%%%%%%%%%%%%%%%%%%%%%%%%%%%%%%%%%%%%%%%%%%%%%%%%%%%%%%%%%%%%%%%%%%

\section{Scheduling variant 2}

    As modelled in Rieck2012.

    \subsection{Hypothesis}

    \begin{itemize}
     \item fixed maintenance duration.
     \item min distance between maintenances that.
     \item each maintenance uses an amount of resource.
     \item objective: minimize the max number of maintenances.
    \end{itemize}

      \subsection{Notations}

    \noindent{
    \begin{tabular}{ll}
        $i \in \mathcal{V} = [\![1, n]\!]$ & Set of maintenance operations.\\
        $(i, j) \in \mathcal{A} = [\![1, v]\!]$ & Set of consecutive (i, j) maintenance operations.\\
        $i = 0$ & Artificial starting node.\\
        $i = n+1$ & Artificial ending node.\\
    \end{tabular}
    }

    \vskip 0.3cm

    The set $\mathcal{A}$ we will construct it by deciding in advance the maintenances $i$ that share the same aircraft.

    \subsection{Parameters}


    \begin{tabular}{ll}
        $p_{i}$ & Duration of maintenance $k$ of resource $i$.\\
        ${\delta}_{ij}$ & Minimum duration between two maintenances $i$ and $j$.\\
        $\mathbf{d}$ & Last period on the horizon.\\
        $r_{i}$ & Resources taken by maintenance $i$.\\
        % $\mathbf{{To}}_{i,k}$ & Minimum starting period for maintenance $k$ of resource $i$.\\
        % $\mathbf{{Ta}}_{i,k}$ & Maximum starting period for maintenance $k$ of resource $i$.\\
    \end{tabular}

    \subsection{Variables}

    \begin{itemize}
     \item $S_i$: the start of maintenance operation $i$.
    \end{itemize}

    \subsection{Model}

    $f(S)$ needs to be defined still. But includes a way to smooth the use of resoures by the activities.

    \begin{align}
        f(S)
    \end{align}

    \begin{align}
        & S_j - S_i \geq \delta_{ij} & (i, j) \in \mathcal{A} \\
        & S_0 = 0 & \\
        & S_{n+1} = \mathbf{d}  \\
        & S_{i} \geq 0, i \in \mathcal{V} \\
    \end{align}

\end{document}



