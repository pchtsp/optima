\documentclass{roadef}
\usepackage{amsmath}
% \usepackage{enumitem}
\usepackage{fancyhdr}
\usepackage{graphicx}  % Required for including images
% \usepackage{setspace}

% \usepackage{fontspec}
% This command is to use simple quotes inside math expressions:
\newcommand{\mq}[1] {`#1\textrm'}
% \setlist{nosep} 

\pagestyle{fancy}
\fancyhf{}  

\lfoot{\includegraphics[height=1.5cm]{img/AFIS_logo_2.jpg}}
% \rfoot{\includegraphics[height=2cm]{img/AFIS_logo_2.jpg}}

% \setstretch{1.2}

\fancypagestyle{plain}{%
  \fancyhf{}
\lfoot{\enspace\includegraphics[height=1.5cm]{img/AFIS_logo_2.jpg}}

}

\begin{document}

% \footskip{1.9pt}
% Le titre du papier
\title{Planning aircraft maintenances for the French Airforce}

% Le titre court
% \def\shorttitle{Titre court}

\author{Franco Peschiera\inst{1}, Alain Haït\inst{1}, Olga Battaïa\inst{1}, Nicolas Dupin\inst{1}}


\institute{ISAE-SUPAERO, Université de Toulouse, France}


\maketitle
% \thispagestyle{empty}


% \vspace{-10em}
\begin{itemize}
	\item[] Franco Peschiera (franco.peschiera@isae-supaero.fr)
	\item[] Directeurs de Thèse: Olga Battaïa, Alain Haït, Nicolas Dupin
	\item[] Institut: ISAE-SUPAERO, Université de Toulouse, France
	\item[] Collaboration industrielle: DGA
\end{itemize}

\keywords{optimization, planning, military, maintenance}

%% keywords here, in the form: keyword \sep keyword

%% MSC codes here, in the form: \MSC code \sep code
%% or \MSC[2008] code \sep code (2000 is the default)

% \end{keyword}

% \end{frontmatter}

%% Title, authors and addresses
% 
\abstract{}

The Flight and Maintenance Planning (FMP) problem deals with the scheduling of maintenance operations and the assignment of flight activities to a fleet of aircraft. It has two main variants: civil (or commercial) and military. We study the latter, the lesser known variant. Previous work include short and medium term planning: \cite{Cho2011,Kozanidis2008,Verhoeff2015,marlow2017optimal,Seif2018}.

A series of $j \in \mathcal{J}$ tasks are planned along a horizon divided into $t \in \mathcal{T}$ periods. Since all tasks are already scheduled, we know time periods $T_j \subset \mathcal{T}$ in which they will be realized. Similarly, all the tasks to be realized in period $t$ are known and defined by set $J_t \subset \mathcal{J}$. Each task requires a certain number $R_{j}$ of resources $i \in \mathcal{I}$ which it employs for a time duration defined by $H_j$ in each of a series of at least $MT_j$ consecutive periods. Set $I_{j} \subset \mathcal{I}$ lists the resources that can be assigned to each task and set $O_i \subset \mathcal{J}$ consists of tasks for which resource $i$ can be used. The number of resources in use for each period $D_t$ is also known.

Each resource can only be assigned to a single task in any given period. These resources suffer from wear and tear and require regular maintenance operations during their lifetime. The need for maintenance is calculated based on two indicators. The first one is called "remaining elapsed time" (or $ret_{it}$). It expresses the amount of time (measured in time periods) after which the resource cannot be used any more and has to undergo a maintenance operation. Its value is calculated for each resource $i$ and each time period $t$. In a similar way, "remaining usage time" (or $rut_{it}$) is used to measure the amount of time that the resource $i$ can be used before needing a maintenance operation at any given period $t$.

Each maintenance operation has a fix duration of $M$ periods and cannot be interrupted: during this time the resource cannot be assigned to any task. After a maintenance operation, a resource restores its remaining elapsed time and remaining usage time to their maximum values $E$ and $H$ respectively.


The objective is to minimize the total number of maintenances and, at the same time, maximize the end status of the resources.     Maintenance capacity is controlled by (\ref{eq:capacity1}). Tasks' resource requirements are defined by (\ref{eq:taskres}). Constraints (\ref{eq:state}) guarantee that a resource can be used only for one task or a maintenance operation at the same period. Constraints (\ref{eq:start1}) control the definition of starting a consecutive assignment: the variable is active if a resource $i$ is assigned a task $j$ in period $t$ and was not assigned this same task in the previous period ($t-1$). Constraints (\ref{eq:start3}) control the minimum duration of a consecutive task assignment. Constraints (\ref{eq:avalaibility-cluster}) guarantee a minimum availability of resources for each cluster $k$. A cluster is defined by the biggest group of resources that is required, exclusively, by at least one task. The minimum and maximum elapsed time are defined by (\ref{eq:ret_min}) and (\ref{eq:ret_max})

\begin{align}
	& \text{Min}\; \sum_{t \in \mathcal{T}, i \in \mathcal{I}} m_{it} \times H - \sum_{i \in \mathcal{I}} rut_{i|\mathcal{T}|} \\
    & \sum_{t' \in \mathcal{T}^{s}_t} \sum_{i \in \mathcal{I}} m_{it'} + N_t \leq M^{max}
      & t \in \mathcal{T} \label{eq:capacity1}\\
    & \sum_{i \in \mathcal{I}_j} a_{jti} = R_j
            & j \in \mathcal{J}, t \in \mathcal{T}_j  \label{eq:taskres}\\
    & \sum_{t' \in \mathcal{T}^{s}_t} m_{it'} + \sum_{j \in \mathcal{J}_t \cap \mathcal{O}_i} a_{jti} \leq 1 
            & t \in \mathcal{T}, i \in \mathcal{I} \label{eq:state}\\
    & a^s_{jti} \geq a_{jti} - a_{j(t-1)i}
            & j \in \mathcal{J}, t \in \mathcal{T}_j, i \in \mathcal{I}_j \label{eq:start1} \\
    & \sum_{t' \in \mathcal{T}^{MT}_{jt}} a^s_{jt'i} \leq a_{jti} 
    & j \in \mathcal{J}, t \in \mathcal{T}_j, i \in \mathcal{I}_j \label{eq:start3}\\
   & \sum_{t' \in \mathcal{T}^{s}_t} \sum_{i \in \mathcal{I}_k} m_{it'} \leq AK_{kt}
    &k \in \mathcal{K}, t \in \mathcal{T} \label{eq:avalaibility-cluster} \\
     & rut_{it} \leq rut_{i(t-1)} + H m_{it} - \sum_{j \in \mathcal{J}_t \cap \mathcal{O}_i} a_{jti} H_j 
     	& t =1, ..., \mathcal{T}, i \in \mathcal{I} \\
    & rut_{i0} = Rut^{Init}_i
           & i \in \mathcal{I} \\
    & rut_{it} \geq H m_{it}
            & t \in \mathcal{T}, i \in \mathcal{I}\\ 
    & rut_{it} \in [0,H]
            & t \in \mathcal{T}, i \in \mathcal{I} \\
    & m_{it'} + m_{it} \leq 1
      & t \in \mathcal{T}, t' \in \mathcal{T}^{m}_t, i \in \mathcal{I}\label{eq:ret_min}\\ 
    & \sum_{t' \in \mathcal{T}^{M}_t} m_{it'} \geq  m_{it}
      & t \in \mathcal{T}, i \in \mathcal{I}\label{eq:ret_max}\\
    & m_{it} = 0
      & t \in \mathcal{T}^{m_{ini}}_i, i \in \mathcal{I} \\
    & \sum_{t \in \mathcal{T}^{M_{ini}}_i} m_{it} \geq  1 
      & i \in \mathcal{I}
\end{align}


\bibliographystyle{plain}
\bibliography{MFMP}

\end{document}