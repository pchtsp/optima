\documentclass{roadef}
\usepackage{amsmath}

% \usepackage{fontspec}
% This command is to use simple quotes inside math expressions:
\usepackage[hyphens,spaces,obeyspaces]{url}
\newcommand{\mq}[1] {`#1\textrm'}

\begin{document}


\title{An alternative MIP formulation for the Military Flight and Maintenance Planning problem}

\author{Franco Peschiera\inst{1}, Alain Ha{\"i}t\inst{1}, Olga Batta{\"i}a\inst{2}, Nicolas Dupin\inst{3}}


\institute{
ISAE-SUPAERO, Universit{\'e} de Toulouse, France \\
\email{\{franco.peschiera,alain.hait\}@isae-supaero.fr}\\
KEDGE Business School, France \\
\email{olga.battaia@kedgebs.com} \\
LRI, université Paris-Saclay, France \\
\email{dupin@lri.fr}
}

\maketitle
\thispagestyle{empty}

\keywords{optimization, planning, military, maintenance}

\section{Introduction}
    The Military Flight and Maintenance Planning Problem assigns missions and schedules maintenance operations (checks) for military aircraft. It has been studied in short, medium and long-term planning horizons (\cite{Cho2011,Kozanidis2008,Verhoeff2015,PeschieraR2018}) and it has been proved NP-Hard in its longterm variant by \cite{Peschiera2019}. The present model is an alternative formulation to that of \cite{Peschiera2019} where mission assignments and maintenance cycles are modeled as start-stop assignment.

    The problem consists in assigning aircraft $i \in \mathcal{I}$ to a set $j \in \mathcal{J}$ of missions while scheduling checks over a time horizon divided into $t \in \mathcal{T}$ periods. Missions have known start and end dates; and requires $R_j$ aircraft to fly $H^\prime_{jtt'}$ when assigned between periods $t$ and $t'$. Each aircraft $i$ can be assigned to only missions $j \in \mathcal{JI}_i$ and is assumed to fly $U^{\prime}_{tt'}$ hours between periods $t$ and $t'$. Each period $t$ has $j \in \mathcal{JT}_t$ active missions.

    Each aircraft $i$ needs a check after $H^{max}$ flight hours or less; has $Rft^{Init}_i$ remaining flight hours at the beginning of the planning horizon; can be assigned its first check during periods $t \in \mathcal{T}^{M_{Init}}_i$ and, if assigned a first check in period $t$, it can have its second check during periods $t \in \mathcal{T}^{M+}_{t}$. $\mathcal{TTT}_t$ is the set of check patterns that make an aircraft be in maintenance in period $t$; $\mathcal{TTJ}_{jt}$ is the set of mission $j$ patterns that make aircraft be in mission in period $t$; $\mathcal{JTT}_{itt'}$ is the set of mission patterns for aircraft $i$ between periods $t$ and $t'$.

\section{Mathematical model}
		
    \begin{tabular}{p{5mm}lp{130mm}}
        $a_{ijtt'}$ & =& 1 if aircraft $i$ starts mission $j$ at period $t$ and finishes at period $t'$, 0 otherwise. \\  
        $m_{itt'}$  & =& 1 if aircraft $i$ starts a check at period $t$ and then starts the next check at period $t'$, 0 otherwise. \\
    \end{tabular}

    The objective is to delay the second check as much as possible or avoid doing it altogether.
    \begin{align}
        & \text{Max}\; 
        \sum_{
                i \in \mathcal{I}, t \in \mathcal{T}^{M_{Init}}_i, t' \in \mathcal{T}^{M+}_t
            } m_{itt'} \times t'
        \label{eq:objective1}
    \end{align}
    
    \begin{align}
        % maximum capacity1:
        & \sum_{i \in \mathcal{I}, (t_1, t_2) \in \mathcal{T}\mathcal{T}\mathcal{T}_{t}} m_{it_1t_2} \leq C^{max} 
                & t \in \mathcal{T} \label{eq:capacity1}\\
        % min assignments:
        & \sum_{i \in \mathcal{IJ}_j, (t_1, t_2) \in \mathcal{T}\mathcal{T}\mathcal{J}_{jt}} a_{ijt_1t_2} \geq R_j
                & j \in \mathcal{J}, t \in \mathcal{TJ}_j  \label{eq:missionres}\\
        % just doing one thing at any given time:
        & \sum_{(t_1, t_2) \in \mathcal{T}\mathcal{T}\mathcal{T}_{t}} m_{it_1t_2} + \sum_{j \in \mathcal{JT}_t \cap \mathcal{JI}_i} \sum_{(t_1, t_2) \in \mathcal{T}\mathcal{T}\mathcal{J}_{jt}} a_{ijt_1t_2} \leq 1 
                & t \in \mathcal{T}, i \in \mathcal{I} \label{eq:state}
    \end{align}

    Constraints (\ref{eq:capacity1}) limit the number of simultaneous checks. Constraints (\ref{eq:missionres}) enforce aircraft mission requirements. Constraints (\ref{eq:state}) restrict each aircraft to at most one assignment each period.

    \begin{align}
        & \sum_{\substack{(j, t, t') \in \\ \mathcal{J}\mathcal{T}\mathcal{T}_{i1t_1}}} a_{ijtt'} H^\prime_{jtt'} + U^{\prime}_{1t_1} \leq Rft^{Init}_i + H^{max} (1 - m_{it_1t_2}) 
            & i \in \mathcal{I}, t_1 \in \mathcal{T}^{M_{Init}}_i, t_2 \in \mathcal{T}^{M+}_{t_1} \label{eq:cycle_hours1}\\
        & \sum_{\substack{(j, t, t') \in \\ \mathcal{J}\mathcal{T}\mathcal{T}_{it_1t_2}}} a_{ijtt'} H^\prime_{jtt'} + U^{\prime}_{t_1t_2} \leq H^{max} + H^{max} (1 - m_{it_1t_2}) 
            & i \in \mathcal{I}, t_1 \in \mathcal{T}^{M_{Init}}_i, t_2 \in \mathcal{T}^{M+}_{t_1} \label{eq:cycle_hours2}\\
        & \sum_{\substack{(j, t, t') \in \\ \mathcal{J}\mathcal{T}\mathcal{T}_{it_2T}}} a_{ijtt'} H^\prime_{jtt'} + U^{\prime}_{t_2T} \leq H^{max} + H^{max} (1 - m_{it_1t_2}) 
            & i \in \mathcal{I}, t_1 \in \mathcal{T}^{M_{Init}}_i, t_2 \in \mathcal{T}^{M+}_{t_1} \label{eq:cycle_hours3}
    \end{align}

    Constraints (\ref{eq:cycle_hours1}) - (\ref{eq:cycle_hours3}) limit the total flight hours of each aircraft before the first check, between checks and after the second check.

    \begin{align}
        & \sum_{t \in \mathcal{T}^{M_{Init}}_i, t' \in \mathcal{T}^{M+}_{t}} m_{itt'} =  1 
          & i \in \mathcal{I}\label{eq:num_maint}
    \end{align}

    Constraints (\ref{eq:num_maint}) require a check pattern assignment for each aircraft.

\section{Conclusions and future work}

    The present work proposes a new model for the Military Flight and Maintenance Problem. The formulation results in better lower bounds (LP-relaxation) compared to \cite{Peschiera2019} and lends itself to pattern prediction forecasting models, which can be used to reduce the size of the model.

% La bibliographie

\bibliographystyle{plain}
% \selectlanguage{english}
\bibliography{./../biblio/MFMP,./../biblio/FrancoPeschiera}

\end{document}
