\documentclass[ignorenonframetext,]{beamer}
\setbeamertemplate{caption}[numbered]
\setbeamertemplate{caption label separator}{: }
\setbeamercolor{caption name}{fg=normal text.fg}
\beamertemplatenavigationsymbolsempty
\usepackage{lmodern}
\usepackage{amssymb,amsmath}
\usepackage{ifxetex,ifluatex}
\usepackage{fixltx2e} % provides \textsubscript
\ifnum 0\ifxetex 1\fi\ifluatex 1\fi=0 % if pdftex
  \usepackage[T1]{fontenc}
  \usepackage[utf8]{inputenc}
\else % if luatex or xelatex
  \ifxetex
    \usepackage{mathspec}
  \else
    \usepackage{fontspec}
  \fi
  \defaultfontfeatures{Ligatures=TeX,Scale=MatchLowercase}
\fi
\usetheme[]{Berlin}
% use upquote if available, for straight quotes in verbatim environments
\IfFileExists{upquote.sty}{\usepackage{upquote}}{}
% use microtype if available
\IfFileExists{microtype.sty}{%
\usepackage{microtype}
\UseMicrotypeSet[protrusion]{basicmath} % disable protrusion for tt fonts
}{}
\newif\ifbibliography
\hypersetup{
            pdftitle={Planification de maintenances des aéronefs militaires},
            pdfborder={0 0 0},
            breaklinks=true}
\urlstyle{same}  % don't use monospace font for urls
\usepackage{longtable,booktabs}
\usepackage{caption}
% These lines are needed to make table captions work with longtable:
\makeatletter
\def\fnum@table{\tablename~\thetable}
\makeatother

% Prevent slide breaks in the middle of a paragraph:
\widowpenalties 1 10000
\raggedbottom

\AtBeginPart{
  \let\insertpartnumber\relax
  \let\partname\relax
  \frame{\partpage}
}
\AtBeginSection{
  \ifbibliography
  \else
    \let\insertsectionnumber\relax
    \let\sectionname\relax
    \frame{\sectionpage}
  \fi
}
\AtBeginSubsection{
  \let\insertsubsectionnumber\relax
  \let\subsectionname\relax
  \frame{\subsectionpage}
}

\setlength{\parindent}{0pt}
\setlength{\parskip}{6pt plus 2pt minus 1pt}
\setlength{\emergencystretch}{3em}  % prevent overfull lines
\providecommand{\tightlist}{%
  \setlength{\itemsep}{0pt}\setlength{\parskip}{0pt}}
\setcounter{secnumdepth}{0}

\title{Planification de maintenances des aéronefs militaires}
\author{Franco Peschiera, Alain Haït, Olga Battaïa}
\institute{ISAE SUPAERO}
\date{June 06, 2018}

\begin{document}
\frame{\titlepage}

\begin{frame}
\tableofcontents[hideallsubsections]
\end{frame}

\section{Introduction}\label{introduction}

\begin{frame}{Planification Thèse}

\includegraphics[width=1\linewidth]{./../../img/gantt_thesis}

\end{frame}

\section{Problème}\label{probleme}

\begin{frame}{Avions}

Une flotte hétérogène d'avions avec les caractéristiques suivantes:

\begin{itemize}[<+->]
\tightlist
\item
  Temps écoulé restant.
\item
  Des heures de vol restant.
\item
  Fonctionnalités.
\end{itemize}

\end{frame}

\begin{frame}{Missions}

Un ensemble de missions à réaliser, chaque une avec:

\begin{itemize}[<+->]
\tightlist
\item
  Dates de début et fin.
\item
  Un besoin mensuelle d'avions.
\item
  Une consommation d'heures par avion-mois affecté.
\item
  Des fonctionnalités requise.
\end{itemize}

\end{frame}

\begin{frame}{Maintenances}

Chaque maintenance suit les règles suivantes:

\begin{itemize}[<+->]
\tightlist
\item
  La \emph{durée est de 6 mois} consécutifs.
\item
  Le nombre maximal \emph{d'heures de vol} entre deux maintenances est
  \emph{1000h}.
\item
  Le nombre maximal de \emph{mois calendrier} entre deux maintenances
  est \emph{60 mois}.
\end{itemize}

\end{frame}

\begin{frame}{États possibles}

En résumé, voici les états logiques possibles d'un aéronef:

\begin{itemize}[<+->]
\tightlist
\item
  En mission.
\item
  En maintenance.
\item
  En stockage.
\item
  Disponible.
\end{itemize}

\end{frame}

\begin{frame}{Objectifs}

\begin{itemize}[<+->]
\tightlist
\item
  Maximiser la disponibilité.
\item
  Minimiser le coût (réduisant le nombre de maintenances).
\end{itemize}

\end{frame}

\begin{frame}{Problème (exemple)}

\includegraphics[width=1\linewidth]{./../../img/calendar}

\end{frame}

\begin{frame}{Contraintes / Règles / Objectifs}

\begin{itemize}[<+->]
\tightlist
\item
  RQ001: heures de missions
\item
  RQ002: aéronefs assignés
\item
  RQ003: besoins de maintenance
\item
  RQ004: durée de maintenance
\item
  RQ005: disponibilité
\item
  RQ006: capacité de maintenance
\item
  RQ007: état final
\item
  RQ008: stockage
\end{itemize}

\end{frame}

\section{État de l'art}\label{etat-de-lart}

\begin{frame}{État de l'art (1)}

\begin{itemize}[<+->]
\tightlist
\item
  FMP: Flight and Maintenance Planning problem.
\item
  En Cho (2011), avions de l'Armée de l'air états-unienne ont été
  affectés à des opérations quotidiens pour minimiser le nombre maximal
  de maintenances.
\item
  En Kozanidis (2008), des heures de vol et des maintenances ont été
  affectées aux avions de la flotte grecque.
\item
  En Verhoeff, Verhagen, and Curran (2015), différents objectifs ont été
  essayé pour les affectations d'heures de vol et maintenances.
\end{itemize}

\end{frame}

\begin{frame}{État de l'art (2)}

Les travaux précédents:

\begin{itemize}[<+->]
\tightlist
\item
  ont étudié de flottes homogènes: nous avons de flottes hétérogènes.
\item
  n'ont pas le mêmes contraintes (stockages, 60 périodes entre
  maintenances) ou les mêmes objectifs (maximiser disponibilité).
\item
  n'affectent pas les avions aux missions.
\end{itemize}

\end{frame}

\section{Modèle}\label{modele}

\begin{frame}{Modèle: fonction objectif}

\begin{align}
    & \text{Min}\; W_1 m_{max} + W_2 u_{max}
\end{align}

\pause

Minimiser le nombre maximal de maintenances et le nombre maximal de
non-disponibilités.

\begin{align}
    &\sum_{t' \in \mathcal{T}^{s}_t} \sum_{i \in \mathcal{I}} m_{it'} + N_t \leq m_{max}
    &t \in \mathcal{T} \\
    &\sum_{t' \in \mathcal{T}^{s}_t} \sum_{i \in \mathcal{I}} m_{it'} + N_t + D_t\leq u_{max}
    &t \in \mathcal{T}
\end{align}

\end{frame}

\begin{frame}{Modèle: contraintes principales}

Besoins des missions et incompatibilités entre les missions et les
maintenances.

\begin{align}
&\sum_{i \in \mathcal{I}_j} a_{jti} = R_j
&j \in \mathcal{J}, t \in \mathcal{T}_j\\
&\sum_{t' \in \mathcal{T}^{s}_t} m_{it'} + \sum_{j \in \mathcal{J}_t \cap \mathcal{O}_i} a_{jti} \leq 1
& t \in \mathcal{T}, i \in \mathcal{I}
\end{align}

\end{frame}

\begin{frame}{Modèle: contraintes de flux}

Calcule d'heures de vol restantes pour chaque avion:

\begin{align}
& rut_{it} \leq rut_{it-1} + H m_{it} - \sum_{j \in \mathcal{J}_t \cap \mathcal{O}_i} a_{jti} H_j & t =1, ..., \mathcal{T}, i \in \mathcal{I}\\
& rut_{i0} = Rut^{Init}_i
        & i \in \mathcal{I}\\
& rut_{it} \geq H m_{it}
        & t \in \mathcal{T}, i \in \mathcal{I}\\
& rut_{it} \in [0,H]
        & t \in \mathcal{T}, i \in \mathcal{I} \\
& \sum_{i \in \mathcal{I}} rut_{it} \geq Rut^{Init}_{sum}
        & t = |\mathcal{T}|
\end{align}

\end{frame}

\section{Résultats}\label{resultats}

\begin{frame}{Expériences}

Des instances ont déjà été testés:

\begin{longtable}[]{@{}lllllll@{}}
\toprule
id & \(\|\mathcal{J}\|\) & \(\|\mathcal{T}\|\) & assign & objective &
time (s) & bound\tabularnewline
\midrule
\endhead
I\_0 & 9 & 11 & 310 & 62.0 & 0.7 & 62.0\tabularnewline
I\_1 & 9 & 21 & 650 & 63.0 & 68.7 & 63.0\tabularnewline
I\_2 & 9 & 31 & 990 & 63.0 & 3600.1 & 62.0\tabularnewline
I\_3 & 9 & 41 & 1249 & 64.0 & 3603.9 & 61.7\tabularnewline
I\_4 & 10 & 11 & 530 & 82.0 & 0.9 & 82.0\tabularnewline
I\_5 & 10 & 21 & 1070 & 83.0 & 144.0 & 83.0\tabularnewline
I\_6 & 10 & 31 & 1610 & 83.0 & 3600.1 & 82.0\tabularnewline
I\_7 & 10 & 41 & 2069 & 84.0 & 3609.1 & 81.8\tabularnewline
I\_8 & 11 & 11 & 1080 & 139.0 & 530.6 & 139.0\tabularnewline
I\_9 & 11 & 21 & 2120 & 149.0 & 3600.0 & 139.9\tabularnewline
\bottomrule
\end{longtable}

\end{frame}

\section{Prochains pas}\label{prochains-pas}

\begin{frame}{Prochains pas}

\begin{itemize}[<+->]
\tightlist
\item
  \textbf{L'amélioration du modèle mathématique}: casser les symétries,
  reformuler les objectifs.
\item
  \textbf{Solutions techniques alternatives}: essayer d'autres méthodes
  de résolution.
\item
  \textbf{Continuer le dialogue avec les utilisateurs}: ajouter des
  nouvelles contraintes, valider les résultats.
\end{itemize}

\end{frame}

\section{Références}\label{references}

\begin{frame}{Références / Questions?}

\hypertarget{refs}{}
\hypertarget{ref-Cho2011}{}
Cho, P. 2011. ``Optimal Scheduling of Fighter Aircraft Maintenance.''
PhD thesis, Massachusetts Institute of Technology.

\hypertarget{ref-Kozanidis2008}{}
Kozanidis, G. 2008. ``A Multiobjective Model for Maximizing Fleet
Availability under the Presence of Flight and Maintenance
Requirements.'' \emph{Journal of Advanced Transportation} 43 (2):
155--82.

\hypertarget{ref-Verhoeff2015}{}
Verhoeff, M., W. J C Verhagen, and R. Curran. 2015. ``Maximizing
operational readiness in military aviation by optimizing flight and
maintenance planning.'' \emph{Transportation Research Procedia} 10
(July). Elsevier B.V.: 941--50.
doi:\href{https://doi.org/10.1016/j.trpro.2015.09.048}{10.1016/j.trpro.2015.09.048}.

\end{frame}

\end{document}
